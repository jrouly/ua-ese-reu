\section{Essential Efficiency and Degree of Automation}

%{{{ Alice3
\subsection*{Alice3}

\begin{tabularx}{\textwidth}{Xcc}
\textbf{Use Case} & \textbf{Required User Actions} & \textbf{User Mental Load}\\
\hline
Open File (\ref{app:euc_open}) & 0 & 0 \\
Use Case 2                     & 0 & 0 \\
Use Case 3                     & 0 & 0
\end{tabularx}
%}}}

%{{{ AudioMulch
\subsection*{AudioMulch}

% Use Case 1 - Open File
% * 1)  Indicate that you want to open a file **
% * 2)  Indicate which file you want to open **
% * 3)  Open the file **

% Use Case 2 - Create Element at specific position
% * 1)  Indicate that you want to create an element
% * 2)  Indicate which element you want to create **
% * 3)  Designate the position at which you want to create the element **
% * 4)  Create the element **

% Use Case 3 - Create two different elements and link them
% * 1)  Indicate that you want to create an element
% * 2)  Indicate which element you want to create **
% * 3)  Designate the position at which you want to create the element **
% * 4)  Create the element **
% * 5)  Indicate that you want to create another element
% * 6)  Indicate which element you want to create **
% * 7)  Designate the position at which you want to create the element **
% * 8)  Create the element **
% * 9)  Designate the first element to be linked **
% * 10) Designate the second element to be linked **
% * 11) Create the link **

\begin{tabularx}{\textwidth}{Xcc}
\textbf{Use Case} & \textbf{Required User Actions} & \textbf{User Mental Load}\\
\hline
Use Case 1                          & 3 & 3 \\
Use Case 2                          & 3 & 3 \\
Use Case 3                          & 9 & 9
\end{tabularx}
%}}}

%{{{ Blender
\subsection*{Blender}

% Use Case 1 - Open File
% * 1)  Indicate that you want to open a file **
% * 2)  Indicate which file you want to open **
% * 3)  Open the file

% Use Case 2 - Create Element at specific position
% * 1)  Indicate that you want to create an element
% * 2)  Indicate which element you want to create **
% * 3)  Designate the position at which you want to create the element **
% * 4)  Create the element **

\begin{tabularx}{\textwidth}{Xcc}
\textbf{Use Case} & \textbf{Required User Actions} & \textbf{User Mental Load}\\
\hline
Open File (\ref{app:euc_open}) & 0 & 0 \\
Use Case 2                     & 0 & 0 \\
Use Case 3                     & 0 & 0
\end{tabularx}
%}}}

%{{{ Cameleon
\subsection*{Cameleon}

\begin{tabularx}{\textwidth}{Xcc}
\textbf{Use Case} & \textbf{Required User Actions} & \textbf{User Mental Load}\\
\hline
Open File (\ref{app:euc_open}) & 0 & 0 \\
Use Case 2                     & 0 & 0 \\
Use Case 3                     & 0 & 0
\end{tabularx}
%}}}

%{{{ Eclipse Modeling Framework
\subsection*{Eclipse Modeling Framework}

\begin{tabularx}{\textwidth}{Xcc}
\textbf{Use Case} & \textbf{Required User Actions} & \textbf{User Mental Load}\\
\hline
Open File (\ref{app:euc_open}) & 0 & 0 \\
Use Case 2                     & 0 & 0 \\
Use Case 3                     & 0 & 0
\end{tabularx}
%}}}

%{{{ GNU Radio Companion
\subsection*{GNU Radio Companion}

\begin{tabularx}{\textwidth}{Xcc}
\textbf{Use Case} & \textbf{Required User Actions} & \textbf{User Mental Load}\\
\hline
Open File (\ref{app:euc_open}) & 0 & 0 \\
Use Case 2                     & 0 & 0 \\
Use Case 3                     & 0 & 0
\end{tabularx}
%}}}

%{{{ Grasshopper 3D
\subsection*{Grasshopper 3D}

% Use Case 1 - Open File
% * 1)  Indicate that you want to open a file **
% * 2)  Indicate which file you want to open **
% * 3)  Open the file **

% Use Case 2 - Create Element at specific position
% * 1)  Indicate that you want to create an element
% * 2)  Indicate which element you want to create **
% * 3)  Designate the position at which you want to create the element **
% * 4)  Create the element **

% Use Case 3 - Create two different elements and link them
% * 1)  Indicate that you want to create an element
% * 2)  Indicate which element you want to create **
% * 3)  Designate the position at which you want to create the element **
% * 4)  Create the element **
% * 5)  Indicate that you want to create another element
% * 6)  Indicate which element you want to create **
% * 7)  Designate the position at which you want to create the element **
% * 8)  Create the element **
% * 9)  Designate the first element to be linked **
% * 10) Designate the second element to be linked **
% * 11) Create the link **

\begin{tabularx}{\textwidth}{Xcc}
\textbf{Use Case} & \textbf{Required User Actions} & \textbf{User Mental Load}\\
\hline
Use Case 1                          & 4 & 3 \\
Use Case 2                          & 3 & 3 \\
Use Case 3                          & 9 & 9
\end{tabularx}
%}}}

%{{{ Max
\subsection*{Max}

\begin{tabularx}{\textwidth}{Xcc}
\textbf{Use Case} & \textbf{Required User Actions} & \textbf{User Mental Load}\\
\hline
Open File (\ref{app:euc_open}) & 0 & 0 \\
Use Case 2                     & 0 & 0 \\
Use Case 3                     & 0 & 0
\end{tabularx}
%}}}

%{{{ MetaEdit+
\subsection*{MetaEdit+}

% Use Case 1 - Open File
% * 1)  Indicate that you want to open a file **
% * 2)  Indicate which file you want to open **
% * 3)  Open the file **

% Use Case 2 - Create Element at specific position
% * 1)  Indicate that you want to create an element
% * 2)  Indicate which element you want to create **
% * 3)  Designate the position at which you want to create the element **
% * 4)  Create the element **

% Use Case 3 - Create two different elements and link them
% * 1)  Indicate that you want to create an element
% * 2)  Indicate which element you want to create **
% * 3)  Designate the position at which you want to create the element **
% * 4)  Create the element **
% * 5)  Indicate that you want to create another element
% * 6)  Indicate which element you want to create **
% * 7)  Designate the position at which you want to create the element **
% * 8)  Create the element **
% * 9)  Designate the first element to be linked **
% * 10) Designate the second element to be linked **
% * 11) Create the link **

\begin{tabularx}{\textwidth}{Xcc}
\textbf{Use Case} & \textbf{Required User Actions} & \textbf{User Mental Load}\\
\hline
Use Case 1                          & 6 & 3 \\
Use Case 2                          & n & 3 \\
Use Case 3                          & 2n + 4 & 9
\end{tabularx}
%}}}

%{{{ MIT AppInventor
\subsection*{MIT AppInventor}

\begin{tabularx}{\textwidth}{Xcc}
\textbf{Use Case} & \textbf{Required User Actions} & \textbf{User Mental Load}\\
\hline
Open File (\ref{app:euc_open}) & 0 & 0 \\
Use Case 2                     & 0 & 0 \\
Use Case 3                     & 0 & 0
\end{tabularx}
%}}}

%{{{ MST Workshop
\subsection*{MST Workshop}

% Use Case 1 - Open File
% * 1)  Indicate that you want to open a file **
% * 2)  Indicate which file you want to open **
% * 3)  Open the file **

% Use Case 2 - Create Element at specific position
% * 1)  Indicate that you want to create an element
% * 2)  Indicate which element you want to create **
% * 3)  Designate the position at which you want to create the element **
% * 4)  Create the element **

% Use Case 3 - Create two different elements and link them
% * 1)  Indicate that you want to create an element
% * 2)  Indicate which element you want to create **
% * 3)  Designate the position at which you want to create the element **
% * 4)  Create the element **
% * 5)  Indicate that you want to create another element
% * 6)  Indicate which element you want to create **
% * 7)  Designate the position at which you want to create the element **
% * 8)  Create the element **
% * 9)  Designate the first element to be linked **
% * 10) Designate the second element to be linked **
% * 11) Create the link

\begin{tabularx}{\textwidth}{Xcc}
\textbf{Use Case} & \textbf{Required User Actions} & \textbf{User Mental Load}\\
\hline
Use Case 1                          & 4 & 3 \\
Use Case 2                          & 3 & 3 \\
Use Case 3                          & 8 & 8
\end{tabularx}
%}}}

%{{{ Piet Creator
\subsection*{Piet Creator}

% Use Case 1 - Open File
% * 1)  Indicate that you want to open a file **
% * 2)  Indicate which file you want to open **
% * 3)  Open the file **

% Use Case 2 - Create Element at specific position
% * 1)  Indicate that you want to create an element
% * 2)  Indicate which element you want to create **
% * 3)  Designate the position at which you want to create the element **
% * 4)  Create the element **

% Use Case 3 - Print an integer
% * 1)  Indicate which color will be the starting color **
% * 2)  Create the color block of size x **
% * 3)  Determine which color represents "push"
% * 4)  Indicate that you will begin to use that color **
% * 5)  Create its color block **
% * 6)  Determine which color represents "out(integer)"
% * 7)  Indicate that you will begin to use that color **
% * 8)  Create its color block **
% * 9)  Determine where any black codels will need to be **
% * 10) Select the color black **
% * 11) Create the black codels **
% * 12) Execute the program **

\begin{tabularx}{\textwidth}{Xcc}
\textbf{Use Case} & \textbf{Required User Actions} & \textbf{User Mental Load}\\
\hline
Use Case 1                          & 3 & 3 \\
Use Case 2                          & 3 & 3 \\
Use Case 3                          & 16 & 10
\end{tabularx}
%}}}

%{{{ Scratch
\subsection*{Scratch}

\begin{tabularx}{\textwidth}{Xcc}
\textbf{Use Case} & \textbf{Required User Actions} & \textbf{User Mental Load}\\
\hline
Open File (\ref{app:euc_open}) & 0 & 0 \\
Use Case 2                     & 0 & 0 \\
Use Case 3                     & 0 & 0
\end{tabularx}
%}}}

%{{{ Simulink
\subsection*{Simulink}

% Use Case 1 - Open File
% * 1)  Indicate that you want to open a file
% * 2)  Indicate which file you want to open
% * 3)  Open the file

% Use Case 2 - Create Element at specific position
% * 1)  Indicate that you want to create an element
% * 2)  Indicate which element you want to create
% * 3)  Designate the position at which you want to create the element
% * 4)  Create the element

% Use Case 3 - Create two different elements and link them
% * 1)  Indicate that you want to create an element
% * 2)  Indicate which element you want to create
% * 3)  Designate the position at which you want to create the element
% * 4)  Create the element
% * 5)  Indicate that you want to create another element
% * 6)  Indicate which element you want to create
% * 7)  Designate the position at which you want to create the element
% * 8)  Create the element
% * 9)  Designate the first element to be linked
% * 10) Designate the second element to be linked
% * 11) Create the link

\begin{tabularx}{\textwidth}{Xcc}
\textbf{Use Case} & \textbf{Required User Actions} & \textbf{User Mental Load}\\
\hline
Use Case 1                          & 0 & 0 \\
Use Case 2                          & 0 & 0 \\
Use Case 3                          & 0 & 0
\end{tabularx}
%}}}

%{{{ Stencyl
\subsection*{Stencyl}

% Use Case 1 - Open File
% * 1)  Indicate that you want to open a file **
% * 2)  Indicate which file you want to open **
% * 3)  Open the file **

% Use Case 2 - Create Element at specific position
% * 1)  Indicate that you want to create an element
% * 2)  Indicate which element you want to create **
% * 3)  Designate the position at which you want to create the element **
% * 4)  Create the element **

% Use Case 3 - Create two different elements and link them
% * 1)  Indicate that you want to create an element
% * 2)  Indicate which element you want to create **
% * 3)  Designate the position at which you want to create the element **
% * 4)  Create the element **
% * 5)  Indicate that you want to create another element
% * 6)  Indicate which element you want to create **
% * 7)  Designate the position at which you want to create the element **
% * 8)  Create the element **
% * 9)  Designate the first element to be linked
% * 10) Designate the second element to be linked
% * 11) Create the link **

\begin{tabularx}{\textwidth}{Xcc}
\textbf{Use Case} & \textbf{Required User Actions} & \textbf{User Mental Load}\\
\hline
Use Case 1                          & 4 & 3 \\
Use Case 2                          & 3 & 3 \\
Use Case 3                          & 6 & 7
\end{tabularx}
%}}}

%{{{ Tersus
\subsection*{Tersus}

% Use Case 1 - Open File
% * 1)  Indicate that you want to open a file **
% * 2)  Indicate which file you want to open **
% * 3)  Open the file **

% Use Case 2 - Create Element at specific position
% * 1)  Indicate that you want to create an element
% * 2)  Indicate which element you want to create **
% * 3)  Designate the position at which you want to create the element **
% * 4)  Create the element **

% Use Case 3 - Create two different elements and link them
% * 1)  Indicate that you want to create an element
% * 2)  Indicate which element you want to create **
% * 3)  Designate the position at which you want to create the element **
% * 4)  Create the element **
% * 5)  Indicate that you want to create another element
% * 6)  Indicate which element you want to create **
% * 7)  Designate the position at which you want to create the element **
% * 8)  Create the element **
% * 9)  Designate the first element to be linked **
% * 10) Designate the second element to be linked **
% * 11) Create the link

\begin{tabularx}{\textwidth}{Xcc}
\textbf{Use Case} & \textbf{Required User Actions} & \textbf{User Mental Load}\\
\hline
Use Case 1                          & 4 & 3 \\
Use Case 2                          & 4 & 3 \\
Use Case 3                          & 11 & 8
\end{tabularx}
%}}}

%{{{ TouchDevelop
\subsection*{TouchDevelop}

% Use Case 1 - Open File
% * 1)  Indicate that you want to open a file
% * 2)  Indicate which file you want to open
% * 3)  Open the file

% Use Case 2 - Create Element at specific position
% * 1)  Indicate that you want to create an element
% * 2)  Indicate which element you want to create
% * 3)  Designate the position at which you want to create the element
% * 4)  Create the element

% Use Case 3 - Print "Hello World"'

\begin{tabularx}{\textwidth}{Xcc}
\textbf{Use Case} & \textbf{Required User Actions} & \textbf{User Mental Load}\\
\hline
Use Case 1                          & 0 & 0 \\
Use Case 2                          & 0 & 0 \\
Use Case 3                          & 0 & 0
\end{tabularx}
%}}}

%{{{ UMLet
\subsection*{UMLet}

\begin{tabularx}{\textwidth}{Xcc}
\textbf{Use Case} & \textbf{Required User Actions} & \textbf{User Mental Load}\\
\hline
Open File (\ref{app:euc_open}) & 0 & 0 \\
Use Case 2                     & 0 & 0 \\
Use Case 3                     & 0 & 0
\end{tabularx}
%}}}

%{{{ VisSim
\subsection*{VisSim}

\begin{tabularx}{\textwidth}{Xcc}
\textbf{Use Case} & \textbf{Required User Actions} & \textbf{User Mental Load}\\
\hline
Open File (\ref{app:euc_open}) & 0 & 0 \\
Use Case 2                     & 0 & 0 \\
Use Case 3                     & 0 & 0
\end{tabularx}
%}}}

%{{{ Visual Paradigm
\subsection*{Visual Paradigm}

% Use Case 1 - Open File
% * 1)  Indicate that you want to open a file **
% * 2)  Indicate which file you want to open **
% * 3)  Open the file **

% Use Case 2 - Create Element at specific position
% * 1)  Indicate that you want to create an element
% * 2)  Indicate which element you want to create **
% * 3)  Designate the position at which you want to create the element **
% * 4)  Create the element **

% Use Case 3 - Create two different elements and link them
% * 1)  Indicate that you want to create an element
% * 2)  Indicate which element you want to create **
% * 3)  Designate the position at which you want to create the element **
% * 4)  Create the element **
% * 5)  Indicate that you want to create another element
% * 6)  Indicate which element you want to create **
% * 7)  Designate the position at which you want to create the element **
% * 8)  Create the element **
% * 9)  Designate the first element to be linked **
% * 10) Designate the second element to be linked **
% * 11) Create the link

\begin{tabularx}{\textwidth}{Xcc}
\textbf{Use Case} & \textbf{Required User Actions} & \textbf{User Mental Load}\\
\hline
Use Case 1                          & 4 & 3 \\
Use Case 2                          & 3 & 3 \\
Use Case 3                          & 10 & 9
\end{tabularx}
%}}}

%{{{ Visual Use Case
\subsection*{Visual Use Case}

% Use Case 1 - Open File
% * 1)  Indicate that you want to open a file
% * 2)  Indicate which file you want to open
% * 3)  Open the file

% Use Case 2 - Create Element at specific position
% * 1)  Indicate that you want to create an element
% * 2)  Indicate which element you want to create
% * 3)  Designate the position at which you want to create the element
% * 4)  Create the element

% Use Case 3 - Create two different elements and link them
% * 1)  Indicate that you want to create an element
% * 2)  Indicate which element you want to create
% * 3)  Designate the position at which you want to create the element
% * 4)  Create the element
% * 5)  Indicate that you want to create another element
% * 6)  Indicate which element you want to create
% * 7)  Designate the position at which you want to create the element
% * 8)  Create the element
% * 9)  Designate the first element to be linked
% * 10) Designate the second element to be linked
% * 11) Create the link

\begin{tabularx}{\textwidth}{Xcc}
\textbf{Use Case} & \textbf{Required User Actions} & \textbf{User Mental Load}\\
\hline
Use Case 1                          & 15 & 3 \\
Use Case 2                          & 20 & 4 \\
Use Case 3                          & 25 & 2
\end{tabularx}
%}}}

%{{{ WebRatio
\subsection*{WebRatio}

% Use Case 1 - Open File
% * 1)  Indicate that you want to open a file
% * 2)  Indicate which file you want to open
% * 3)  Open the file

% Use Case 2 - Create Element at specific position
% * 1)  Indicate that you want to create an element
% * 2)  Indicate which element you want to create
% * 3)  Designate the position at which you want to create the element
% * 4)  Create the element

% Use Case 3 - Create two different elements and link them
% * 1)  Indicate that you want to create an element
% * 2)  Indicate which element you want to create
% * 3)  Designate the position at which you want to create the element
% * 4)  Create the element
% * 5)  Indicate that you want to create another element
% * 6)  Indicate which element you want to create
% * 7)  Designate the position at which you want to create the element
% * 8)  Create the element
% * 9)  Designate the first element to be linked
% * 10) Designate the second element to be linked
% * 11) Create the link

\begin{tabularx}{\textwidth}{Xcc}
\textbf{Use Case} & \textbf{Required User Actions} & \textbf{User Mental Load}\\
\hline
Use Case 1                          & 15 & 3 \\
Use Case 2                          & 20 & 4 \\
Use Case 3                          & 25 & 2
\end{tabularx}
%}}}

%{{{ YAWL
\subsection*{YAWL}

\begin{tabularx}{\textwidth}{Xcc}
\textbf{Use Case} & \textbf{Required User Actions} & \textbf{User Mental Load}\\
\hline
Open File (\ref{app:euc_open}) & 0 & 0 \\
Use Case 2                     & 0 & 0 \\
Use Case 3                     & 0 & 0
\end{tabularx}
%}}}
