\section{Essential Efficiency and Degree of Automation}

%{{{ Alice3
\subsection*{Alice3}

% Use Case 1 - Open File
% * 1)  Indicate that you want to open a file
% * 2)  Indicate which file you want to open
% * 3)  Open the file

% Use Case 2 - Create Element at specific position
%   1)  Indicate that you want to create an element
% * 2)  Indicate which element you want to create
% * 3)  Designate the position at which you want to create the element
% * 4)  Create the element

% Use Case 3 - Create two different elements and link them
%   1)  Indicate that you want to create an element
% * 2)  Indicate which element you want to create
% * 3)  Designate the position at which you want to create the element
% * 4)  Create the element
%   5)  Indicate that you want to create another element
% * 6)  Indicate which element you want to create
% * 7)  Designate the position at which you want to create the element
% * 8)  Create the element
%   9)  Indicate that you want to begin linking elements
%   10) Indicate type of connection to create.
%   11) Designate the first element to be linked
%   12) Designate the second element to be linked
% * 13) Create the link

\begin{tabularx}{\textwidth}{Xcc}
\textbf{Use Case} & \textbf{Required User Actions} & \textbf{User Mental Load}\\
\hline
Open File (\ref{app:euc_open})                       & 7  & 3 \\
Create Element (\ref{app:euc_create})                & 6  & 3 \\
Create and Link Elements (\ref{app:euc_create_link}) & 12 & 7
\end{tabularx}
%}}}

%{{{ AudioMulch
\subsection*{AudioMulch}

% Use Case 1 - Open File
% * 1)  Indicate that you want to open a file
% * 2)  Indicate which file you want to open
% * 3)  Open the file

% Use Case 2 - Create Element at specific position
%   1)  Indicate that you want to create an element
% * 2)  Indicate which element you want to create
% * 3)  Designate the position at which you want to create the element
% * 4)  Create the element

% Use Case 3 - Create two different elements and link them
%   1)  Indicate that you want to create an element
% * 2)  Indicate which element you want to create
% * 3)  Designate the position at which you want to create the element
% * 4)  Create the element
%   5)  Indicate that you want to create another element
% * 6)  Indicate which element you want to create
% * 7)  Designate the position at which you want to create the element
% * 8)  Create the element
%   9)  Indicate that you want to begin linking elements
%   10) Indicate type of connection to create
% * 11) Designate the first element to be linked
% * 12) Designate the second element to be linked
% * 13) Create the link

\begin{tabularx}{\textwidth}{Xcc}
\textbf{Use Case} & \textbf{Required User Actions} & \textbf{User Mental Load}\\
\hline
Open File (\ref{app:euc_open})                       & 3 & 3 \\
Create Element (\ref{app:euc_create})                & 3 & 3 \\
Create and Link Elements (\ref{app:euc_create_link}) & 9 & 9
\end{tabularx}
%}}}

%{{{ Blender
\subsection*{Blender}

% Use Case 1 - Open File
% * 1)  Indicate that you want to open a file
% * 2)  Indicate which file you want to open
% * 3)  Open the file

% Use Case 2 - Create Element at specific position
% * 1)  Indicate that you want to create an element
% * 2)  Indicate which element you want to create
% * 3)  Designate the position at which you want to create the element
% * 4)  Create the element

% Use Case 5 - Create Element at specific position and transform vertex
% * 1)  Indicate that you want to create an element
% * 2)  Indicate which element you want to create
% * 3)  Designate the position at which you want to create the element
% * 4)  Create the element
% * 5)  Indicate that you want to edit the element
% * 6)  Indicate that you want to transform a vertex
% * 7)  Select a vertex to transform
% * 8)  Indicate the type of transformation to perform
% * 9)  Perform transformation
% * 10) Commit changes to element

\begin{tabularx}{\textwidth}{Xcc}
\textbf{Use Case} & \textbf{Required User Actions} & \textbf{User Mental Load}\\
\hline
Open File (\ref{app:euc_open})        & 4 & 3 \\
Create Element (\ref{app:euc_create}) & 6 & 4 \\
Create Element and Transform (\ref{app:euc_create_transform}) & 12 & 10
\end{tabularx}
%}}}

%{{{ Cameleon
\subsection*{Cameleon}

% Use Case 1 - Open File
% * 1)  Indicate that you want to open a file
% * 2)  Indicate which file you want to open
% * 3)  Open the file

% Use Case 2 - Create Element at specific position
%   1)  Indicate that you want to create an element
% * 2)  Indicate which element you want to create
% * 3)  Designate the position at which you want to create the element
% * 4)  Create the element

% Use Case 3 - Create two different elements and link them
%   1)  Indicate that you want to create an element
% * 2)  Indicate which element you want to create
% * 3)  Designate the position at which you want to create the element
% * 4)  Create the element
%   5)  Indicate that you want to create another element
% * 6)  Indicate which element you want to create
% * 7)  Designate the position at which you want to create the element
% * 8)  Create the element
% * 9)  Indicate that you want to begin linking elements
%   10) Indicate type of connection to create.
% * 11) Designate the first element to be linked
% * 12) Designate the second element to be linked
% * 13) Create the link

\begin{tabularx}{\textwidth}{Xcc}
\textbf{Use Case} & \textbf{Required User Actions} & \textbf{User Mental Load}\\
\hline
Open File (\ref{app:euc_open})                       & 4 & 3 \\
Create Element (\ref{app:euc_create})                & 3 & 3 \\
Create and Link Elements (\ref{app:euc_create_link}) & 9 & 10
\end{tabularx}
%}}}

%{{{ Eclipse Modeling Framework
\subsection*{Eclipse Modeling Framework}

\begin{tabularx}{\textwidth}{Xcc}
\textbf{Use Case} & \textbf{Required User Actions} & \textbf{User Mental Load}\\
\hline
Open File (\ref{app:euc_open}) & {\color{red}0} & {\color{red}0} \\
Use Case 2                     & {\color{red}0} & {\color{red}0} \\
Use Case 3                     & {\color{red}0} & {\color{red}0}
\end{tabularx}
%}}}

%{{{ GNU Radio Companion
\subsection*{GNU Radio Companion}

% Use Case 1 - Open File
% * 1)  Indicate that you want to open a file
% * 2)  Indicate which file you want to open
% * 3)  Open the file

% Use Case 2 - Create Element at specific position
%   1)  Indicate that you want to create an element
% * 2)  Indicate which element you want to create
% * 3)  Designate the position at which you want to create the element
% * 4)  Create the element

% Use Case 3 - Create two different elements and link them
%   1)  Indicate that you want to create an element
% * 2)  Indicate which element you want to create
% * 3)  Designate the position at which you want to create the element
% * 4)  Create the element
%   5)  Indicate that you want to create another element
% * 6)  Indicate which element you want to create
% * 7)  Designate the position at which you want to create the element
% * 8)  Create the element
%   9)  Indicate that you want to begin linking elements
%   10) Indicate type of connection to create.
% * 11) Designate the first element to be linked
% * 12) Designate the second element to be linked
%   13) Create the link

\begin{tabularx}{\textwidth}{Xcc}
\textbf{Use Case} & \textbf{Required User Actions} & \textbf{User Mental Load}\\
\hline
Open File (\ref{app:euc_open})                       & 3 & 3 \\
Create Element (\ref{app:euc_create})                & 3 & 3 \\
Create and Link Elements (\ref{app:euc_create_link}) & 5 & 8
\end{tabularx}
%}}}

%{{{ Grasshopper 3D
\subsection*{Grasshopper 3D}

% Use Case 1 - Open File
% * 1)  Indicate that you want to open a file
% * 2)  Indicate which file you want to open
% * 3)  Open the file

% Use Case 2 - Create Element at specific position
%   1)  Indicate that you want to create an element
% * 2)  Indicate which element you want to create
% * 3)  Designate the position at which you want to create the element
% * 4)  Create the element

% Use Case 3 - Create two different elements and link them
%   1)  Indicate that you want to create an element
% * 2)  Indicate which element you want to create
% * 3)  Designate the position at which you want to create the element
% * 4)  Create the element
%   5)  Indicate that you want to create another element
% * 6)  Indicate which element you want to create
% * 7)  Designate the position at which you want to create the element
% * 8)  Create the element
%   9)  Indicate that you want to begin linking elements
%   10) Indicate type of connection to create
% * 11) Designate the first element to be linked
% * 12) Designate the second element to be linked
% * 13) Create the link

\begin{tabularx}{\textwidth}{Xcc}
\textbf{Use Case} & \textbf{Required User Actions} & \textbf{User Mental Load}\\
\hline
Use Case 1                          & 4 & 3 \\
Use Case 2                          & 3 & 3 \\
Use Case 3                          & 9 & 9
\end{tabularx}
%}}}

%{{{ Max
\subsection*{Max}

\begin{tabularx}{\textwidth}{Xcc}
\textbf{Use Case} & \textbf{Required User Actions} & \textbf{User Mental Load}\\
\hline
Open File (\ref{app:euc_open}) & {\color{red}0} & {\color{red}0} \\
Use Case 2                     & {\color{red}0} & {\color{red}0} \\
Use Case 3                     & {\color{red}0} & {\color{red}0}
\end{tabularx}
%}}}

%{{{ MetaEdit+
\subsection*{MetaEdit+}

% Use Case 1 - Open File
% * 1)  Indicate that you want to open a file
% * 2)  Indicate which file you want to open
% * 3)  Open the file

% Use Case 2 - Create Element at specific position
%   1)  Indicate that you want to create an element
% * 2)  Indicate which element you want to create
% * 3)  Designate the position at which you want to create the element
% * 4)  Create the element

% Use Case 3 - Create two different elements and link them
%   1)  Indicate that you want to create an element
% * 2)  Indicate which element you want to create
% * 3)  Designate the position at which you want to create the element
% * 4)  Create the element
%   5)  Indicate that you want to create another element
% * 6)  Indicate which element you want to create
% * 7)  Designate the position at which you want to create the element
% * 8)  Create the element
%   9)  Indicate that you want to begin linking elements
%   10) Indicate type of connection to create
% * 11) Designate the first element to be linked
% * 12) Designate the second element to be linked
% * 13) Create the link

\begin{tabularx}{\textwidth}{Xcc}
\textbf{Use Case} & \textbf{Required User Actions} & \textbf{User Mental Load}\\
\hline
Use Case 1                          & 6 & 3 \\
Use Case 2                          & n & 3 \\
Use Case 3                          & 2n + 4 & 9
\end{tabularx}
%}}}

%{{{ MIT AppInventor
\subsection*{MIT AppInventor}

% Use Case 1 - Open File
% * 1)  Indicate that you want to open a file
% * 2)  Indicate which file you want to open
% * 3)  Open the file

% Use Case 2 - Create Element at specific position
%   1)  Indicate that you want to create an element
% * 2)  Indicate which element you want to create
% * 3)  Designate the position at which you want to create the element
% * 4)  Create the element

% Use Case 3 - Create two different elements and link them
%   1)  Indicate that you want to create an element
% * 2)  Indicate which element you want to create
% * 3)  Designate the position at which you want to create the element
% * 4)  Create the element
%   5)  Indicate that you want to create another element
% * 6)  Indicate which element you want to create
% * 7)  Designate the position at which you want to create the element
% * 8)  Create the element
%   9)  Indicate that you want to begin linking elements
%   10) Indicate type of connection to create
%   11) Designate the first element to be linked
%   12) Designate the second element to be linked
% * 13) Create the link

\begin{tabularx}{\textwidth}{Xcc}
\textbf{Use Case} & \textbf{Required User Actions} & \textbf{User Mental Load}\\
\hline
Open File (\ref{app:euc_open})                       & 3 & 3 \\
Create Element (\ref{app:euc_create})                & 3 & 3 \\
Create and Link Elements (\ref{app:euc_create_link}) & 6 & 7
\end{tabularx}
%}}}

%{{{ MST Workshop
\subsection*{MST Workshop}

% Use Case 1 - Open File
% * 1)  Indicate that you want to open a file
% * 2)  Indicate which file you want to open
% * 3)  Open the file

% Use Case 2 - Create Element at specific position
%   1)  Indicate that you want to create an element
% * 2)  Indicate which element you want to create
% * 3)  Designate the position at which you want to create the element
% * 4)  Create the element

% Use Case 3 - Create two different elements and link them
%   1)  Indicate that you want to create an element
% * 2)  Indicate which element you want to create
% * 3)  Designate the position at which you want to create the element
% * 4)  Create the element
%   5)  Indicate that you want to create another element
% * 6)  Indicate which element you want to create
% * 7)  Designate the position at which you want to create the element
% * 8)  Create the element
%   9)  Indicate that you want to begin linking elements
%   10) Indicate type of connection to create
% * 11) Designate the first element to be linked
% * 12) Designate the second element to be linked
%   13) Create the link

\begin{tabularx}{\textwidth}{Xcc}
\textbf{Use Case} & \textbf{Required User Actions} & \textbf{User Mental Load}\\
\hline
Use Case 1                          & 4 & 3 \\
Use Case 2                          & 3 & 3 \\
Use Case 3                          & 8 & 8
\end{tabularx}
%}}}

%{{{ Piet Creator
\subsection*{Piet Creator}

% Use Case 1 - Open File
% * 1)  Indicate that you want to open a file
% * 2)  Indicate which file you want to open
% * 3)  Open the file

% Use Case 2 - Create Element at specific position
%   1)  Indicate that you want to create an element
% * 2)  Indicate which element you want to create
% * 3)  Designate the position at which you want to create the element
% * 4)  Create the element

% Use Case 3 - Print an integer
% * 1)  Indicate which color will be the starting color
% * 2)  Create the color block of size x
%   3)  Determine which color represents "push"
% * 4)  Indicate that you will begin to use that color
% * 5)  Create its color block
%   6)  Determine which color represents "out(integer)"
% * 7)  Indicate that you will begin to use that color
% * 8)  Create its color block
% * 9)  Determine where any black codels will need to be
% * 10) Select the color black
% * 11) Create the black codels
% * 12) Execute the program

\begin{tabularx}{\textwidth}{Xcc}
\textbf{Use Case} & \textbf{Required User Actions} & \textbf{User Mental Load}\\
\hline
Use Case 1                          & 3 & 3 \\
Use Case 2                          & 3 & 3 \\
Use Case 3                          & 16 & 10
\end{tabularx}
%}}}

%{{{ Scratch
\subsection*{Scratch}

% Use Case 1 - Open File
% * 1)  Indicate that you want to open a file
% * 2)  Indicate which file you want to open
% * 3)  Open the file

% Use Case 2 - Create Element at specific position
%   1)  Indicate that you want to create an element
% * 2)  Indicate which element you want to create
% * 3)  Designate the position at which you want to create the element
% * 4)  Create the element

% Use Case 3 - Create two different elements and link them
%   1)  Indicate that you want to create an element
% * 2)  Indicate which element you want to create
% * 3)  Designate the position at which you want to create the element
% * 4)  Create the element
%   5)  Indicate that you want to create another element
% * 6)  Indicate which element you want to create
% * 7)  Designate the position at which you want to create the element
% * 8)  Create the element
%   9)  Indicate that you want to begin linking elements
%   10) Indicate type of connection to create
%   11) Designate the first element to be linked
%   12) Designate the second element to be linked
% * 13) Create the link

\begin{tabularx}{\textwidth}{Xcc}
\textbf{Use Case} & \textbf{Required User Actions} & \textbf{User Mental Load}\\
\hline
Open File (\ref{app:euc_open})                       & 5 & 3 \\
Create Element (\ref{app:euc_create})                & 3 & 3 \\
Create and Link Elements (\ref{app:euc_create_link}) & 6 & 7
\end{tabularx}
%}}}

%{{{ Simulink
\subsection*{Simulink}

% Use Case 1 - Open File
%   1)  Indicate that you want to open a file
%   2)  Indicate which file you want to open
%   3)  Open the file

% Use Case 2 - Create Element at specific position
%   1)  Indicate that you want to create an element
%   2)  Indicate which element you want to create
%   3)  Designate the position at which you want to create the element
%   4)  Create the element

% Use Case 3 - Create two different elements and link them
%   1)  Indicate that you want to create an element
%   2)  Indicate which element you want to create
%   3)  Designate the position at which you want to create the element
%   4)  Create the element
%   5)  Indicate that you want to create another element
%   6)  Indicate which element you want to create
%   7)  Designate the position at which you want to create the element
%   8)  Create the element
%   9)  Indicate that you want to begin linking elements
%   10) Indicate type of connection to create
%   11) Designate the first element to be linked
%   12) Designate the second element to be linked
%   13) Create the link

\begin{tabularx}{\textwidth}{Xcc}
\textbf{Use Case} & \textbf{Required User Actions} & \textbf{User Mental Load}\\
\hline
Use Case 1                          & {\color{red}0} & {\color{red}0} \\
Use Case 2                          & {\color{red}0} & {\color{red}0} \\
Use Case 3                          & {\color{red}0} & {\color{red}0}
\end{tabularx}
%}}}

%{{{ Stencyl
\subsection*{Stencyl}

% Use Case 1 - Open File
% * 1)  Indicate that you want to open a file
% * 2)  Indicate which file you want to open
% * 3)  Open the file

% Use Case 2 - Create Element at specific position
%   1)  Indicate that you want to create an element
% * 2)  Indicate which element you want to create
% * 3)  Designate the position at which you want to create the element
% * 4)  Create the element

% Use Case 3 - Create two different elements and link them
%   1)  Indicate that you want to create an element
% * 2)  Indicate which element you want to create
% * 3)  Designate the position at which you want to create the element
% * 4)  Create the element
%   5)  Indicate that you want to create another element
% * 6)  Indicate which element you want to create
% * 7)  Designate the position at which you want to create the element
% * 8)  Create the element
%   9)  Indicate that you want to begin linking elements
%   10) Indicate type of connection to create
%   11) Designate the first element to be linked
%   12) Designate the second element to be linked
% * 13) Create the link

\begin{tabularx}{\textwidth}{Xcc}
\textbf{Use Case} & \textbf{Required User Actions} & \textbf{User Mental Load}\\
\hline
Use Case 1                          & 4 & 3 \\
Use Case 2                          & 3 & 3 \\
Use Case 3                          & 6 & 7
\end{tabularx}
%}}}

%{{{ Tersus
\subsection*{Tersus}

% Use Case 1 - Open File
% * 1)  Indicate that you want to open a file
% * 2)  Indicate which file you want to open
% * 3)  Open the file

% Use Case 2 - Create Element at specific position
%   1)  Indicate that you want to create an element
% * 2)  Indicate which element you want to create
% * 3)  Designate the position at which you want to create the element
% * 4)  Create the element

% Use Case 3 - Create two different elements and link them
%   1)  Indicate that you want to create an element
% * 2)  Indicate which element you want to create
% * 3)  Designate the position at which you want to create the element
% * 4)  Create the element
%   5)  Indicate that you want to create another element
% * 6)  Indicate which element you want to create
% * 7)  Designate the position at which you want to create the element
% * 8)  Create the element
% * 9)  Indicate that you want to begin linking elements
%   10) Indicate type of connection to create
% * 11) Designate the first element to be linked
% * 12) Designate the second element to be linked
%   13) Create the link

\begin{tabularx}{\textwidth}{Xcc}
\textbf{Use Case} & \textbf{Required User Actions} & \textbf{User Mental Load}\\
\hline
Use Case 1                          & 4 & 3 \\
Use Case 2                          & 4 & 3 \\
Use Case 3                          & 11 & 9
\end{tabularx}
%}}}

%{{{ TouchDevelop
\subsection*{TouchDevelop}

% Use Case 1 - Open File
%   1)  Indicate that you want to open a file
%   2)  Indicate which file you want to open
%   3)  Open the file

% Use Case 2 - Create Element at specific position
%   1)  Indicate that you want to create an element
%   2)  Indicate which element you want to create
%   3)  Designate the position at which you want to create the element
%   4)  Create the element

% Use Case 3 - Print "Hello World"'

\begin{tabularx}{\textwidth}{Xcc}
\textbf{Use Case} & \textbf{Required User Actions} & \textbf{User Mental Load}\\
\hline
Use Case 1                          & {\color{red}0} & {\color{red}0} \\
Use Case 2                          & {\color{red}0} & {\color{red}0} \\
Use Case 3                          & {\color{red}0} & {\color{red}0}
\end{tabularx}
%}}}

%{{{ UMLet
\subsection*{UMLet}

% Use Case 1 - Open File
% * 1)  Indicate that you want to open a file
% * 2)  Indicate which file you want to open
% * 3)  Open the file

% Use Case 2 - Create Element at specific position
%   1)  Indicate that you want to create an element
% * 2)  Indicate which element you want to create
% * 3)  Designate the position at which you want to create the element
% * 4)  Create the element

% Use Case 3 - Create two different elements and link them
%   1)  Indicate that you want to create an element
% * 2)  Indicate which element you want to create
% * 3)  Designate the position at which you want to create the element
% * 4)  Create the element
%   5)  Indicate that you want to create another element
% * 6)  Indicate which element you want to create
% * 7)  Designate the position at which you want to create the element
% * 8)  Create the element
%   9)  Indicate that you want to begin linking elements
%   10) Indicate type of connection to create
% * 11) Designate the first element to be linked
% * 12) Designate the second element to be linked
% * 13) Create the link

\begin{tabularx}{\textwidth}{Xcc}
\textbf{Use Case} & \textbf{Required User Actions} & \textbf{User Mental Load}\\
\hline
Open File (\ref{app:euc_open})                       & 4  & 3 \\
Create Element (\ref{app:euc_create})                & 3  & 3 \\
Create and Link Elements (\ref{app:euc_create_link}) & 15 & 9
\end{tabularx}
%}}}

%{{{ VioletUML
\subsection*{VioletUML}

% Use Case 1 - Open File
% * 1)  Indicate that you want to open a file
% * 2)  Indicate which file you want to open
% * 3)  Open the file

% Use Case 2 - Create Element at specific position
%   1)  Indicate that you want to create an element
% * 2)  Indicate which element you want to create
% * 3)  Designate the position at which you want to create the element
% * 4)  Create the element

% Use Case 3 - Create two different elements and link them
%   1)  Indicate that you want to create an element
% * 2)  Indicate which element you want to create
% * 3)  Designate the position at which you want to create the element
% * 4)  Create the element
%   5)  Indicate that you want to create another element
% * 6)  Indicate which element you want to create
% * 7)  Designate the position at which you want to create the element
% * 8)  Create the element
% * 9)  Indicate that you want to begin linking elements
% * 10) Indicate type of connection to create.
% * 11) Designate the first element to be linked
% * 12) Designate the second element to be linked
% * 13) Create the link

\begin{tabularx}{\textwidth}{Xcc}
\textbf{Use Case} & \textbf{Required User Actions} & \textbf{User Mental Load}\\
\hline
Open File (\ref{app:euc_open})                       & 4  & 3 \\
Create Element (\ref{app:euc_create})                & 3  & 3 \\
Create and Link Elements (\ref{app:euc_create_link}) & 11 & 11
\end{tabularx}
%}}}

%{{{ VisSim
\subsection*{VisSim}

\begin{tabularx}{\textwidth}{Xcc}
\textbf{Use Case} & \textbf{Required User Actions} & \textbf{User Mental Load}\\
\hline
Open File (\ref{app:euc_open}) & {\color{red}0} & {\color{red}0} \\
Use Case 2                     & {\color{red}0} & {\color{red}0} \\
Use Case 3                     & {\color{red}0} & {\color{red}0}
\end{tabularx}
%}}}

%{{{ Visual Paradigm
\subsection*{Visual Paradigm}

% Use Case 1 - Open File
% * 1)  Indicate that you want to open a file
% * 2)  Indicate which file you want to open
% * 3)  Open the file

% Use Case 2 - Create Element at specific position
%   1)  Indicate that you want to create an element
% * 2)  Indicate which element you want to create
% * 3)  Designate the position at which you want to create the element
% * 4)  Create the element

% Use Case 3 - Create two different elements and link them
%   1)  Indicate that you want to create an element
% * 2)  Indicate which element you want to create
% * 3)  Designate the position at which you want to create the element
% * 4)  Create the element
%   5)  Indicate that you want to create another element
% * 6)  Indicate which element you want to create
% * 7)  Designate the position at which you want to create the element
% * 8)  Create the element
% * 9)  Indicate that you want to begin linking elements
%   10) Indicate type of connection to create
% * 11) Designate the first element to be linked
% * 12) Designate the second element to be linked
%   13) Create the link

\begin{tabularx}{\textwidth}{Xcc}
\textbf{Use Case} & \textbf{Required User Actions} & \textbf{User Mental Load}\\
\hline
Use Case 1                          & 4 & 3 \\
Use Case 2                          & 3 & 3 \\
Use Case 3                          & 10 & 9
\end{tabularx}
%}}}

%{{{ Visual Use Case
\subsection*{Visual Use Case}

% Use Case 1 - Open File
% * 1)  Indicate that you want to open a file
% * 2)  Indicate which file you want to open
% * 3)  Open the file

% Use Case 2 - Create Element at specific position
%   1)  Indicate that you want to create an element
% * 2)  Indicate which element you want to create
% * 3)  Designate the position at which you want to create the element
% * 4)  Create the element

% Use Case 3 - Create two different elements and link them
%   1)  Indicate that you want to create an element
% * 2)  Indicate which element you want to create
% * 3)  Designate the position at which you want to create the element
% * 4)  Create the element
%   5)  Indicate that you want to create another element
% * 6)  Indicate which element you want to create
% * 7)  Designate the position at which you want to create the element
% * 8)  Create the element
%   9)  Indicate that you want to begin linking elements
%   10) Indicate type of connection to create
%   11) Designate the first element to be linked
%   12) Designate the second element to be linked
%   13) Create the link

\begin{tabularx}{\textwidth}{Xcc}
\textbf{Use Case} & \textbf{Required User Actions} & \textbf{User Mental Load}\\
\hline
Use Case 1                          & 4 & 3 \\
Use Case 2                          & 4 & 3 \\
Use Case 3                          & 8 & 6
\end{tabularx}
%}}}

%{{{ WebRatio
\subsection*{WebRatio}

% Use Case 1 - Open File
% * 1)  Indicate that you want to open a file
% * 2)  Indicate which file you want to open
% * 3)  Open the file

% Use Case 2 - Create Element at specific position
%   1)  Indicate that you want to create an element
% * 2)  Indicate which element you want to create
% * 3)  Designate the position at which you want to create the element
% * 4)  Create the element

% Use Case 3 - Create two different elements and link them
%   1)  Indicate that you want to create an element
% * 2)  Indicate which element you want to create
% * 3)  Designate the position at which you want to create the element
% * 4)  Create the element
%   5)  Indicate that you want to create another element
% * 6)  Indicate which element you want to create
% * 7)  Designate the position at which you want to create the element
% * 8)  Create the element
% * 9)  Indicate that you want to begin linking elements
%   10) Indicate type of connection to create
% * 11) Designate the first element to be linked
% * 12) Designate the second element to be linked
%   13) Create the link

\begin{tabularx}{\textwidth}{Xcc}
\textbf{Use Case} & \textbf{Required User Actions} & \textbf{User Mental Load}\\
\hline
Use Case 1                          & 4 & 3 \\
Use Case 2                          & 3 & 3 \\
Use Case 3                          & 10 & 9
\end{tabularx}
%}}}

%{{{ YAWL
\subsection*{YAWL}

% Use Case 1 - Open File
% * 1)  Indicate that you want to open a file
% * 2)  Indicate which file you want to open
% * 3)  Open the file

% Use Case 2 - Create Element at specific position
%   1)  Indicate that you want to create an element
% * 2)  Indicate which element you want to create
% * 3)  Designate the position at which you want to create the element
% * 4)  Create the element

% Use Case 3 - Create two different elements and link them
%   1)  Indicate that you want to create an element
% * 2)  Indicate which element you want to create
% * 3)  Designate the position at which you want to create the element
% * 4)  Create the element
%   5)  Indicate that you want to create another element
% * 6)  Indicate which element you want to create
% * 7)  Designate the position at which you want to create the element
% * 8)  Create the element
% * 9)  Indicate that you want to begin linking elements
%   10) Indicate type of connection to create.
% * 11) Designate the first element to be linked
% * 12) Designate the second element to be linked
%   13) Create the link

\begin{tabularx}{\textwidth}{Xcc}
\textbf{Use Case} & \textbf{Required User Actions} & \textbf{User Mental Load}\\
\hline
Open File (\ref{app:euc_open})                       & 3 & 3 \\
Create Element (\ref{app:euc_create})                & 3 & 3 \\
Create and Link Elements (\ref{app:euc_create_link}) & 9 & 9
\end{tabularx}
%}}}
