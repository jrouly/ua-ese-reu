\section{Essential Use Cases}

\subsection{Open File}
\label{app:euc_open}

Opening a file is a simple, fundamental Essential Use Case for any
\ac{ide}, visual or otherwise. It refers to fetching a single file from
storage and loading it into the active editing workspace.

\begin{enumerate}
  \item Indicate that you want to open a file.
  \item Indicate which file you want to open.
  \item Open the file.
\end{enumerate}


\subsection{Create Element}

Visual languages are primarly composed of networks of linked elements.
Therefore, creating one such element is a fundamental Essential Use Case.
This Use Case requires that the element must be created and placed at a
specific, user defined location.

\begin{enumerate}
  \item Indicate that you want to create an element.
  \item Indicate which element you want to create.
  \item Designate the position at which you want to create the element.
  \item Create the element.
\end{enumerate}


\subsection{Create and Link Elements}

Visual Languages also allow elements to share links to one another. This
Use Case requires two unique elements to be created and placed by the user,
and then linked together in a meaningful fashion.

\begin{enumerate}
  \item Indicate that you want to create an element.
  \item Indicate which element you want to create.
  \item Designate the position at which you want to create the element.
  \item Create the element.
  \item Indicate that you want to create another element.
  \item Indicate which element you want to create.
  \item Designate the position at which you want to create the element.
  \item Create the element.
  \item Designate the first element to be linked.
  \item Designate the second element to be linked.
  \item Create the link.
\end{enumerate}
