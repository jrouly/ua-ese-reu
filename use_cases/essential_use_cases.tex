\section{Essential Use Cases}

%{{{ Open File EUC C.1
\subsection{Open File}
\label{app:euc_open}

Opening a file is a simple, fundamental Essential Use Case for any
\ac{ide}, visual or otherwise. It refers to fetching a single file from
storage and loading it into the active editing workspace.

\begin{enumerate}
  \item Indicate that you want to open a file.
  \item Indicate which file you want to open.
  \item Open the file.
\end{enumerate}
%}}}

%{{{ Create Element EUC C.2
\subsection{Create Element}
\label{app:euc_create}

Visual languages are primarily composed of networks of linked elements.
Therefore, creating one such element is a fundamental Essential Use Case.
This Use Case requires that the element must be created and placed at a
specific, user defined location.

\begin{enumerate}
  \item Indicate that you want to create an element.
  \item Indicate which element you want to create.
  \item Designate the position at which you want to create the element.
  \item Create the element.
\end{enumerate}
%}}}

%{{{ Create and Link Elements EUC C.3
\subsection{Create and Link Elements}
\label{app:euc_create_link}

Visual Languages also allow elements to share links to one another. This
Use Case requires two unique elements to be created and placed by the user,
and then linked together in a meaningful fashion.

\begin{enumerate}
  \item Indicate that you want to create an element.
  \item Indicate which element you want to create.
  \item Designate the position at which you want to create the element.
  \item Create the element.
  \item Indicate that you want to create another element.
  \item Indicate which element you want to create.
  \item Designate the position at which you want to create the element.
  \item Create the element.
  \item Designate the first element to be linked.
  \item Designate the second element to be linked.
  \item Create the link.
\end{enumerate}
%}}}

%{{{ Print an Integer EUC C.4
\subsection{Print an Integer}
\label{app:euc_print_integer}

\begin{enumerate}
  \item Indicate which color will be the starting color.
  \item Create the color block of size x.
  \item Determine which color represents ``push''.
  \item Indicate that you will begin to use that color.
  \item Create its color block.
  \item Determine which color represents ``out (integer)''.
  \item Indicate that you will begin to use that color.
  \item Create its color block.
  \item Determine where any black codels will need to be.
  \item Select the color black.
  \item Create the black codels.
  \item Execute the program.
\end{enumerate}
%}}}

%{{{ Create Element and Transform EUC C.5
\subsection{Create Element and Transform}
\label{app:euc_create_transform}

\begin{enumerate}
  \item Indicate that you want to create an element.
  \item Indicate which element you want to create.
  \item Designate the position at which you want to create the element.
  \item Create the element.
  \item Indicate that you want to edit the element.
  \item Select a vertex to transform.
  \item Perform transformation.
  \item Save changes to element.
\end{enumerate}
%}}}

%{{{ Print a Character EUC C.6
\subsection{Print a Character}
\label{app:euc_print_character}

\begin{enumerate}
  \item Designate the position at which you want to create an element.
  \item Indicate that you want to create a character.
  \item Designate the character that you want to create.
  \item Create the character.
  \item Designate the position at which you want to place ``post to wall''.
  \item Indicate that you want to create a new command.
  \item Indicate that you want to create ``post to wall''.
  \item Create the command.
  \item Run the program.
\end{enumerate}
%}}}
