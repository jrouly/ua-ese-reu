\section{Discussion}
\label{sec:discussion}

%A visual depiction of the features studied is given in
%\Fig{fig:featuremodel} as a feature diagram model. Observed relationships
%between these features are discussed in this section.

Agreement on visual features indicates a common theme among studied \acp{ide}
. The feature that exhibited the most agreement between \acp{ide} was the 
number of available perspectives, with 13 \acp{ide} containing 
only a single one. The next most agreed upon features are 
interface visual richness and language visual richness which both share 9 
\acp{ide} that support only four visual richness variables for either 
interface or language. Note that \acp{ide} do not necessarily employ the same 
number of visual richness variables for interface as well as supported 
language.

The prevalence of a single perspective in most studied \acp{ide}
indicates a tendency toward simplicity of user control. By introducing
multiple perspectives, an \ac{ide} offers the user a richer set of
controls and wider variety of views on the model, but also adds to the
user's mental load in keeping track of new details.
%A single perspective, as is most common among sampled \acp{ide}, indicates a 
%single level of user control as well as a visually and mentally simple design 
%paradigm.

As indicated by Moody~\cite{moody2009} a larger magnitude of visual
richness variables correlates to a more visually discriminating and thus
rapidly understandable interface. With most \acp{ide} settling on four
visual richness variables employed, the common theme among sampled
\acp{ide} is to a mid-range value. It is possible that too many visual
richness features may be considered ``junk'', as per Tufte's warning~\cite{Tufte2001}.
%But around four distinct variables, the common theme in
%the sampled \acp{ide}, might be enough to discriminate tools and yet not
%too much to become junk.

Many \acp{ide} would be able to benefit through the implementation of
simple, positive features like the addition of visual richness variables.
Additionally, only 11 \acp{ide} implement context
sensitive menus. Even simple convenience features, like the
ability to search through available tools, are common in less than half of
studied \acp{ide}. This lack of simple, positive features actually tends to
detract from the overall quality and usability of an interface.

Some \ac{ide} features can be related to one another. The
collected data suggests that interface and essential efficiency are
significantly related.  A Pearson's product-moment correlation test
indicates that the two variables share a correlation coefficient of $0.556$
with $p=0.003893$. Neither interface nor essential efficiency significantly
correlate with visual clutter however, with $p>0.4$ for each.  The
relationship between interface and essential efficiency is expected, not
only due to similarity of the metrics use for each, but because of the
underlying concepts. The more automation or efficiency an interface
directly offers to its users, the more efficiency is
provided to the user as a mental load deduction. That is, without the need
to focus on details of implementation in the interface, the user is free to
concern himself with other manually controlled details. These
manually controlled details are emphasized as more important in the
\ac{ide} designers by virtue of not being automated.

Overall, the relationships determined by this initial study indicate an
emphasis on managing operator mental load in visual \ac{ide} design. Many
sampled \acp{ide} offer a number of visual variables to the user providing
increased visual discrimination for ease of use. Additionally, the
relationship found between interface and essential efficiency hints that
interface design plays directly into the mangitude of user mental load.

%This set of data, along with the accompanying formalizations of visual
%features, is a starting point for future work in the field. As discussed in
%the conclusion, further analytical work using the reported data as well as
%any expanded data set must be performed to fully understand the scope of
%this feature set's impact on usability and suitability.
