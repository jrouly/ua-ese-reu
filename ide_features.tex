\documentclass[letterpaper,10pt]{article}

%{{{ Package includes =========================================================

\usepackage{acro} % for defining acronyms
\usepackage{enumitem} % for custom, richer list environments
\usepackage{fullpage} % use 1in margins
\usepackage[numbers]{natbib} % for useful bibliography tools
\usepackage{amsmath} % better mathematical formatting
\usepackage{tabularx} % tables with column stretching

%}}}

%{{{ Acronym Definitions ======================================================

\DeclareAcronym{ide}{
  short        = IDE ,
  long         = Integrated Development Environment ,
  class        = abbrev
}

%}}}

%{{{ Feature description list definitions =====================================

\newlist{AlignedDesc}{description}{2}
\setlist[AlignedDesc]{leftmargin=10em,style=nextline}

%}}}

\title{\textbf{IDE Features and Comparison}}
\author{Jean Michel Rouly \and Jonathan Orbeck}
\date{\today}

\begin{document}

\maketitle

\tableofcontents
\clearpage

%{{{ Section: Audience ========================================================
\section{Audience}
\label{sec:audience}

The \textbf{Audience} of an \ac{ide} refers generally to the target
populations for which the \ac{ide} is developed.

%{{{ Subsection: Domain
\subsection{Domain}
\label{subsec:domain}

\begin{AlignedDesc}
  \item[Abbreviation] \texttt{Domain}

  \item[Variable Type] Nominal

  \item[Description] The \textbf{Domain} of an \ac{ide} refers to the
  \textit{domain} or field of knowledge under which the interface falls.
  That is, the field or fields for which the interface is primarily used,
  if any. Example values might include \texttt{General}, \texttt{Modeling},
  \texttt{Software}, \textit{etc}.

  \item[Accepted Values]

  \begin{AlignedDesc}
    \item[General] This \ac{ide} is general purpose. It can be applied to
    or used within any number of different fields to the same effect.
    \item[\textellipsis] Any custom field is allowed.
  \end{AlignedDesc}

\end{AlignedDesc}
%}}}

%{{{ Subsection: Skill Level
\subsection{Skill Level}
\label{subsec:skill}

\begin{AlignedDesc}
  \item[Abbreviation] \texttt{Skill}

  \item[Variable Type] Nominal

  \item[Description] The \textbf{Skill Level} of an \ac{ide} describes what
  skill level within the given \textbf{Domain} is expected of its users.

  \item[Accepted Values]

  \begin{AlignedDesc}
    \item[Novice] The user is expected to have little to no experience in
    the domain. The \ac{ide} is designed to be user friendly and welcoming
    to new users with no former domain knowledge.
    \item[Intermediate] The user is expected to have a medium level of
    experience in the domain. The \ac{ide} may offer some advanced
    functionality, but still be welcoming to newer users.
    \item[Expert] The user is expected to have a high level of experience
    in the domain. The \ac{ide} offers highly sophisticated functionality
    that the user is expected to understand without further instruction.
    \item[General] The \ac{ide} offers a powerful set of advanced
    functionality while maintaining an overall level of accessibility and
    ease of use.
  \end{AlignedDesc}

\end{AlignedDesc}
%}}}

%}}}

% {{{ Section: Chrome =========================================================
\section{Chrome}
\label{sec:chrome}

The \textbf{Chrome} of an \ac{ide} is the sum total of user interface
components external to the workspace. This includes every tool, menu,
button, or other user interface component not contained within the active
workspace area.

%{{{ Subsection: Common Features
\subsection{Common Features}
\label{subsec:features}

\begin{AlignedDesc}
  \item[Abbreviation] \texttt{Features}

  \item[Variable Type] Compound

  \item[Description] Many \acp{ide} provide support for a set of common
  operations. \citeauthor{murphy2006} define a list of the top 10 most
  frequently used \ac{ide} features, which are reproduced here as boolean
  sub-variables. \cite{murphy2006} Each sub-variable indicates whether the
  \ac{ide} in question supports the use of that particular feature.

  \item[Components]

  \begin{AlignedDesc}
    \item[Delete] Deletion of a syntactic element in the workspace
    \item[Save] Save and export a model to a storage media
    \item[Paste] Duplicate a syntactic element in the workspace from the
    paste buffer
    \item[Content Assist] Provide suggestions or completion for elements
    \item[Copy] Place syntactic element from the workspace into paste
    buffer
    \item[Undo] Undo the user's most recent action
    \item[Cut] Place a syntactic element from the workspace into paste
    buffer, and remove from workspace
    \item[Refresh] Load contents of workspace and interface dashboard
    elements from storage media and update display if necessary
    \item[Show View] Open and display a new tool in the interface
  \end{AlignedDesc}

\end{AlignedDesc}
%}}}

%{{{ Subsection: Context Sensitive Tools
\subsection{Context Sensitive Tools}
\label{subsec:context}

\begin{AlignedDesc}
  \item[Abbreviation] \texttt{Context}

  \item[Variable Type] Boolean

  \item[Description] Any interface component which changes visibly or is
  generated anew depending on the context of selected element(s) within the
  workspace is a \textbf{Context Sensitive Tool}.

  \item[Accepted Values]

  \begin{AlignedDesc}
    \item[Yes] The \ac{ide} supports context sensitive tools.
    \item[No] The \ac{ide} does not support context sensitive tools.
  \end{AlignedDesc}

\end{AlignedDesc}
%}}}

%{{{ Subsection: Degree of Interface Visual Richness
\subsection{Degree of Interface Visual Richness}
\label{subsec:toolrichness}

\begin{AlignedDesc}
  \item[Abbreviation] \texttt{ToolRichness}

  \item[Variable Type] Compound

  \item[Description] The \textbf{Degree of Interface Visual Richness}
  describes the extent to which an interface utilizes visual variables to
  increase visual discriminability of available tools. The component
  variables measured are a composite of several authors' research,
  including elements from visual language design. Each component variable
  is a boolean measure determining whether or not the described visual
  variable is used in the interface to distinguish available tools.

  \item[Components]

  \begin{AlignedDesc}
    \item[Icons] Images contained in a border of a standard size and shape
    represent distinct actions or tools. \cite{costagliola2002}
    \cite{moody2009}
    \item[Shape] Distinct shapes indicate different tools or classes of
    tool. \cite{moody2009}
    \item[Size] Different tool sizes indicate distinct tools or classes
    of tool. \cite{moody2009}
    \item[Color] Color is used to indicate distinct tools or classes of
    tool. \cite{moody2009}
    \item[Text] Text (or typographic variation) is used to identify or
    distinguish tools or classes of tool. \cite{moody2009}
    \item[Organizational Coherence] Components with related purpose are
    visually grouped in the interface. \cite{constantine1996}
    \item[Texture] Shading or shadows are used to modify tools or to
    distinguish between distinct tools or classes of tool. \cite{moody2009}
    \item[Brightness] The brightness of a color (ie. its perceived
    luminosity) is used to indicate a difference between tools or classes
    of tool. \cite{moody2009}
  \end{AlignedDesc}

\end{AlignedDesc}
%}}}

%{{{ Subsection: Multiplicity of Perspectives
\subsection{Multiplicity of Perspectives}
\label{subsec:perspectives}

\begin{AlignedDesc}
  \item[Abbreviation] \texttt{Perspectives}

  \item[Variable Type] Ratio

  \item[Description] The number of available predefined interface
  perspectives available to the user. A perspective is defined as a visual
  configuration of the available tools in the interface chrome and elements
  in the workspace for the purposes of accomplishing a distinct task by
  means of a distinct process.

  \item[Range] $[1 , \infty)$

\end{AlignedDesc}
%}}}

%{{{ Subsection: Object Properties Window
\subsection{Object Properties Window}
\label{subsec:properties}

\begin{AlignedDesc}
  \item[Abbreviation] \texttt{Properties}

  \item[Variable Type] Nominal

  \item[Description] The \textbf{Object Properties Window} is any interface
  component that displays the properties of an element in the workspace.
  This component generally allows modification of element properties as
  well.

  \item[Accepted Values]

  \begin{AlignedDesc}
    \item[Omnipresent] A single property dialog or window is always
    present. The contents may update contextually.
    \item[Manual] The system requires user interaction to present the
    properties window.
    \item[None] No properties window is ever presented for workspace
    elements.
  \end{AlignedDesc}

\end{AlignedDesc}
%}}}

%{{{ Subsection: Searchable Toolspace
\subsection{Searchable Toolspace}
\label{subsec:searchable}

\begin{AlignedDesc}
  \item[Abbreviation] \texttt{Searchable}

  \item[Variable Type] Boolean

  \item[Description] The \textit{toolspace} is the total set of available
  tools, components, or actions that the \ac{ide} offers to the user. If an
  \ac{ide} employs a \textbf{Searchable Toolspace}, it allows the user to
  search through this toolspace by name or keywords.

  \item[Accepted Values]

  \begin{AlignedDesc}
    \item[Yes] The \ac{ide} supports a searchable toolspace.
    \item[No] The \ac{ide} does not support a searchable toolspace.
  \end{AlignedDesc}

\end{AlignedDesc}
%}}}

%{{{ Subsection: Toolbar Styles [incomplete]
\subsection{Toolbar Styles}
\label{subsec:toolstyle}

\begin{AlignedDesc}
  \item[Abbreviation] \texttt{ToolStyle}

  \item[Variable Type]

  \item[Description]

  \item[Accepted Values]

  \begin{AlignedDesc}
    \item[\textellipsis]
  \end{AlignedDesc}

\end{AlignedDesc}
%}}}

%{{{ Subsection: Visual Clutter [incomplete]
\subsection{Visual Clutter}
\label{subsec:clutter}

\begin{AlignedDesc}
  \item[Abbreviation] \texttt{Clutter}

  \item[Variable Type]

  \item[Description]

  \item[Accepted Values]

  \begin{AlignedDesc}
    \item[\textellipsis]
  \end{AlignedDesc}

\end{AlignedDesc}
%}}}

%}}}

% {{{ Section: Human Interface ================================================
\section{Human Interface}
\label{sec:humaninterface}

\textbf{Human Interface} components of an \ac{ide} include aspects of the
interface that describe how a human user interfaces with the \ac{ide},
either mechanically (\textit{eg.} through physical devices and media) or
mentally (\textit{eg.} the mental load required of the user to operate the
\ac{ide}).

%{{{ Subsection: Degree of Automation [incomplete]
\subsection{Degree of Automation}
\label{subsec:automation}

\begin{AlignedDesc}
  \item[Abbreviation] \texttt{Automation}

  \item[Variable Type] Continuous

  \item[Description] The \textbf{Degree of Automation} of an \ac{ide}
  measures the level to which the system automates tasks for the user. It
  is calculated as the ratio of the number of tasks the user needs to
  perform manually (representing operator mental load) to the number of
  tasks that compose an essential use case.
%
  \begin{align*}
    \frac{\text{One thing}}{\text{Other thing}}
  \end{align*}

\end{AlignedDesc}
%}}}

%{{{ Subsection: Essential Efficiency [incomplete]
\subsection{Essential Efficiency}
\label{subsec:efficiency}

\begin{AlignedDesc}
  \item[Abbreviation] \texttt{Efficiency}

  \item[Variable Type]

  \item[Description]

  \item[Accepted Values]

  \begin{AlignedDesc}
    \item[\textellipsis]
  \end{AlignedDesc}

\end{AlignedDesc}
%}}}

%{{{ Subsection: Keyboard Use [incomplete]
\subsection{Keyboard Use}
\label{subsec:keyboard}

\begin{AlignedDesc}
  \item[Abbreviation] \texttt{Keyboard}

  \item[Variable Type]

  \item[Description]

  \item[Accepted Values]

  \begin{AlignedDesc}
    \item[\textellipsis]
  \end{AlignedDesc}

\end{AlignedDesc}
%}}}

%{{{ Subsection: Mode of Element Creation [incomplete]
\subsection{Mode of Element Creation}
\label{subsec:mode}

\begin{AlignedDesc}
  \item[Abbreviation] \texttt{Mode}

  \item[Variable Type]

  \item[Description]

  \item[Accepted Values]

  \begin{AlignedDesc}
    \item[\textellipsis]
  \end{AlignedDesc}

\end{AlignedDesc}
%}}}

%{{{ Subsection: Tertiary Interface Devices [incomplete]
\subsection{Tertiary Interface Devices}
\label{subsec:devices}

\begin{AlignedDesc}
  \item[Abbreviation] \texttt{Devices}

  \item[Variable Type]

  \item[Description]

  \item[Accepted Values]

  \begin{AlignedDesc}
    \item[\textellipsis]
  \end{AlignedDesc}

\end{AlignedDesc}
%}}}

%}}}

% {{{ Section: Integration ====================================================
\section{Integration}
\label{sec:integration}

%{{{ Subsection: Allowed Relations Indicated [incomplete]
\subsection{Allowed Relations Indicated}
\label{subsec:relations}

\begin{AlignedDesc}
  \item[Abbreviation] \texttt{Relations}

  \item[Variable Type]

  \item[Description]

  \item[Accepted Values]

  \begin{AlignedDesc}
    \item[\textellipsis]
  \end{AlignedDesc}

\end{AlignedDesc}
%}}}

%{{{ Subsection: Output Generation Style [incomplete]
\subsection{Output Generation Style}
\label{subsec:output}

\begin{AlignedDesc}
  \item[Abbreviation] \texttt{Output}

  \item[Variable Type]

  \item[Description]

  \item[Accepted Values]

  \begin{AlignedDesc}
    \item[\textellipsis]
  \end{AlignedDesc}

\end{AlignedDesc}
%}}}

%{{{ Subsection: Syntax Enforcement [incomplete]
\subsection{Syntax Enforcement}
\label{subsec:syntax}

\begin{AlignedDesc}
  \item[Abbreviation] \texttt{Syntax}

  \item[Variable Type]

  \item[Description]

  \item[Accepted Values]

  \begin{AlignedDesc}
    \item[\textellipsis]
  \end{AlignedDesc}

\end{AlignedDesc}
%}}}

%}}}

% {{{ Section: Language Syntax ================================================
\section{Language Syntax}
\label{sec:languagesyntax}

%{{{ Subsection: Complexity Management [incomplete]
\subsection{Complexity Management}
\label{subsec:complexity}

\begin{AlignedDesc}
  \item[Abbreviation] \texttt{Complexity}

  \item[Variable Type]

  \item[Description]

  \item[Accepted Values]

  \begin{AlignedDesc}
    \item[\textellipsis]
  \end{AlignedDesc}

\end{AlignedDesc}
%}}}

%{{{ Subsection: Connection Style [incomplete]
\subsection{Connection Style}
\label{subsec:connection}

\begin{AlignedDesc}
  \item[Abbreviation] \texttt{Connection}

  \item[Variable Type]

  \item[Description]

  \item[Accepted Values]

  \begin{AlignedDesc}
    \item[\textellipsis]
  \end{AlignedDesc}

\end{AlignedDesc}
%}}}

%{{{ Subsection: Degree of Language Visual Richness [incomplete]
\subsection{Degree of Language Visual Richness}
\label{subsec:languagerichness}

\begin{AlignedDesc}
  \item[Abbreviation] \texttt{LanguageRichness}

  \item[Variable Type]

  \item[Description]

  \item[Accepted Values]

  \begin{AlignedDesc}
    \item[\textellipsis]
  \end{AlignedDesc}

\end{AlignedDesc}
%}}}

%}}}

\section{Essential Efficiency and Degree of Automation}

\subsection{Alice3}

% Use Case 1 - Open File
% * 1)  Indicate that you want to open a file
% * 2)  Indicate which file you want to open
% * 3)  Open the file

\begin{tabular*}{\textwidth}{lcc}
\textbf{Use Case} & \textbf{Required User Actions} & \textbf{User Mental Load}\\
\hline
Create Element at Specific Position & 15 & 3 \\
Use Case 2                          & 20 & 4 \\
Use Case 3                          & 25 & 2
\end{tabular*}

\subsection{AudioMulch}

% Use Case 1 - Open File
% * 1)  Indicate that you want to open a file **
% * 2)  Indicate which file you want to open **
% * 3)  Open the file

% Use Case 2 - Create Element at specific position
% * 1)  Indicate that you want to create an element
% * 2)  Indicate which element you want to create **
% * 3)  Designate the position at which you want to create the element **
% * 4)  Create the element **

% Use Case 3 - Create two different elements and link them
% * 1)  Indicate that you want to create an element
% * 2)  Indicate which element you want to create **
% * 3)  Designate the position at which you want to create the element **
% * 4)  Create the element
% * 5)  Indicate that you want to create another element
% * 6)  Indicate which element you want to create
% * 7)  Designate the position at which you want to create the element
% * 8)  Create the element
% * 9)  Designate the first element to be linked
% * 10) Designate the second element to be linked
% * 11) Create the link

\begin{tabular*}{\textwidth}{lcc}
\textbf{Use Case} & \textbf{Required User Actions} & \textbf{User Mental Load}\\
\hline
Use Case 1                          & 2 & 2 \\
Use Case 2                          & 3 & 1 \\
Use Case 3                          & 25 & 2
\end{tabular*}

\subsection{Grasshopper 3D}

% Use Case 1 - Open File
% * 1)  Indicate that you want to open a file
% * 2)  Indicate which file you want to open
% * 3)  Open the file

% Use Case 2 - Create Element at specific position
% * 1)  Indicate that you want to create an element
% * 2)  Indicate which element you want to create
% * 3)  Designate the position at which you want to create the element
% * 4)  Create the element

% Use Case 3 - Create two different elements and link them
% * 1)  Indicate that you want to create an element
% * 2)  Indicate which element you want to create
% * 3)  Designate the position at which you want to create the element
% * 4)  Create the element
% * 5)  Indicate that you want to create another element
% * 6)  Indicate which element you want to create
% * 7)  Designate the position at which you want to create the element
% * 8)  Create the element
% * 9)  Designate the first element to be linked
% * 10) Designate the second element to be linked
% * 11) Create the link

\begin{tabular*}{\textwidth}{lcc}
\textbf{Use Case} & \textbf{Required User Actions} & \textbf{User Mental Load}\\
\hline
Use Case 1                          & 15 & 3 \\
Use Case 2                          & 20 & 4 \\
Use Case 3                          & 25 & 2
\end{tabular*}

\subsection{MetaEdit+}

% Use Case 1 - Open File
% * 1)  Indicate that you want to open a file
% * 2)  Indicate which file you want to open
% * 3)  Open the file

% Use Case 2 - Create Element at specific position
% * 1)  Indicate that you want to create an element
% * 2)  Indicate which element you want to create
% * 3)  Designate the position at which you want to create the element
% * 4)  Create the element

% Use Case 3 - Create two different elements and link them
% * 1)  Indicate that you want to create an element
% * 2)  Indicate which element you want to create
% * 3)  Designate the position at which you want to create the element
% * 4)  Create the element
% * 5)  Indicate that you want to create another element
% * 6)  Indicate which element you want to create
% * 7)  Designate the position at which you want to create the element
% * 8)  Create the element
% * 9)  Designate the first element to be linked
% * 10) Designate the second element to be linked
% * 11) Create the link

\begin{tabular*}{\textwidth}{lcc}
\textbf{Use Case} & \textbf{Required User Actions} & \textbf{User Mental Load}\\
\hline
Use Case 1                          & 15 & 3 \\
Use Case 2                          & 20 & 4 \\
Use Case 3                          & 25 & 2
\end{tabular*}

\subsection{MST Workshop}

% Use Case 1 - Open File
% * 1)  Indicate that you want to open a file
% * 2)  Indicate which file you want to open
% * 3)  Open the file

% Use Case 2 - Create Element at specific position
% * 1)  Indicate that you want to create an element
% * 2)  Indicate which element you want to create
% * 3)  Designate the position at which you want to create the element
% * 4)  Create the element

% Use Case 3 - Create two different elements and link them
% * 1)  Indicate that you want to create an element
% * 2)  Indicate which element you want to create
% * 3)  Designate the position at which you want to create the element
% * 4)  Create the element
% * 5)  Indicate that you want to create another element
% * 6)  Indicate which element you want to create
% * 7)  Designate the position at which you want to create the element
% * 8)  Create the element
% * 9)  Designate the first element to be linked
% * 10) Designate the second element to be linked
% * 11) Create the link

\begin{tabular*}{\textwidth}{lcc}
\textbf{Use Case} & \textbf{Required User Actions} & \textbf{User Mental Load}\\
\hline
Use Case 1                          & 15 & 3 \\
Use Case 2                          & 20 & 4 \\
Use Case 3                          & 25 & 2
\end{tabular*}

\subsection{Piet Creator}

% Use Case 1 - Open File
% * 1)  Indicate that you want to open a file
% * 2)  Indicate which file you want to open
% * 3)  Open the file

% Use Case 2 - Create Element at specific position
% * 1)  Indicate that you want to create an element
% * 2)  Indicate which element you want to create
% * 3)  Designate the position at which you want to create the element
% * 4)  Create the element

% Use Case 3 - Create two different elements and link them
% * 1)  Indicate that you want to create an element
% * 2)  Indicate which element you want to create
% * 3)  Designate the position at which you want to create the element
% * 4)  Create the element
% * 5)  Indicate that you want to create another element
% * 6)  Indicate which element you want to create
% * 7)  Designate the position at which you want to create the element
% * 8)  Create the element
% * 9)  Designate the first element to be linked
% * 10) Designate the second element to be linked
% * 11) Create the link

\begin{tabular*}{\textwidth}{lcc}
\textbf{Use Case} & \textbf{Required User Actions} & \textbf{User Mental Load}\\
\hline
Use Case 1                          & 15 & 3 \\
Use Case 2                          & 20 & 4 \\
Use Case 3                          & 25 & 2
\end{tabular*}

\subsection{Simulink}

% Use Case 1 - Open File
% * 1)  Indicate that you want to open a file
% * 2)  Indicate which file you want to open
% * 3)  Open the file

% Use Case 2 - Create Element at specific position
% * 1)  Indicate that you want to create an element
% * 2)  Indicate which element you want to create
% * 3)  Designate the position at which you want to create the element
% * 4)  Create the element

% Use Case 3 - Create two different elements and link them
% * 1)  Indicate that you want to create an element
% * 2)  Indicate which element you want to create
% * 3)  Designate the position at which you want to create the element
% * 4)  Create the element
% * 5)  Indicate that you want to create another element
% * 6)  Indicate which element you want to create
% * 7)  Designate the position at which you want to create the element
% * 8)  Create the element
% * 9)  Designate the first element to be linked
% * 10) Designate the second element to be linked
% * 11) Create the link

\begin{tabular*}{\textwidth}{lcc}
\textbf{Use Case} & \textbf{Required User Actions} & \textbf{User Mental Load}\\
\hline
Use Case 1                          & 15 & 3 \\
Use Case 2                          & 20 & 4 \\
Use Case 3                          & 25 & 2
\end{tabular*}

\subsection{Stencyl}

% Use Case 1 - Open File
% * 1)  Indicate that you want to open a file
% * 2)  Indicate which file you want to open
% * 3)  Open the file

% Use Case 2 - Create Element at specific position
% * 1)  Indicate that you want to create an element
% * 2)  Indicate which element you want to create
% * 3)  Designate the position at which you want to create the element
% * 4)  Create the element

% Use Case 3 - Create two different elements and link them
% * 1)  Indicate that you want to create an element
% * 2)  Indicate which element you want to create
% * 3)  Designate the position at which you want to create the element
% * 4)  Create the element
% * 5)  Indicate that you want to create another element
% * 6)  Indicate which element you want to create
% * 7)  Designate the position at which you want to create the element
% * 8)  Create the element
% * 9)  Designate the first element to be linked
% * 10) Designate the second element to be linked
% * 11) Create the link

\begin{tabular*}{\textwidth}{lcc}
\textbf{Use Case} & \textbf{Required User Actions} & \textbf{User Mental Load}\\
\hline
Use Case 1                          & 15 & 3 \\
Use Case 2                          & 20 & 4 \\
Use Case 3                          & 25 & 2
\end{tabular*}

\subsection{Tersus}

% Use Case 1 - Open File
% * 1)  Indicate that you want to open a file
% * 2)  Indicate which file you want to open
% * 3)  Open the file

% Use Case 2 - Create Element at specific position
% * 1)  Indicate that you want to create an element
% * 2)  Indicate which element you want to create
% * 3)  Designate the position at which you want to create the element
% * 4)  Create the element

% Use Case 3 - Create two different elements and link them
% * 1)  Indicate that you want to create an element
% * 2)  Indicate which element you want to create
% * 3)  Designate the position at which you want to create the element
% * 4)  Create the element
% * 5)  Indicate that you want to create another element
% * 6)  Indicate which element you want to create
% * 7)  Designate the position at which you want to create the element
% * 8)  Create the element
% * 9)  Designate the first element to be linked
% * 10) Designate the second element to be linked
% * 11) Create the link

\begin{tabular*}{\textwidth}{lcc}
\textbf{Use Case} & \textbf{Required User Actions} & \textbf{User Mental Load}\\
\hline
Use Case 1                          & 15 & 3 \\
Use Case 2                          & 20 & 4 \\
Use Case 3                          & 25 & 2
\end{tabular*}

\subsection{TouchDevelop}

% Use Case 1 - Open File
% * 1)  Indicate that you want to open a file
% * 2)  Indicate which file you want to open
% * 3)  Open the file

% Use Case 2 - Create Element at specific position
% * 1)  Indicate that you want to create an element
% * 2)  Indicate which element you want to create
% * 3)  Designate the position at which you want to create the element
% * 4)  Create the element

% Use Case 3 - Create two different elements and link them
% * 1)  Indicate that you want to create an element
% * 2)  Indicate which element you want to create
% * 3)  Designate the position at which you want to create the element
% * 4)  Create the element
% * 5)  Indicate that you want to create another element
% * 6)  Indicate which element you want to create
% * 7)  Designate the position at which you want to create the element
% * 8)  Create the element
% * 9)  Designate the first element to be linked
% * 10) Designate the second element to be linked
% * 11) Create the link

\begin{tabular*}{\textwidth}{lcc}
\textbf{Use Case} & \textbf{Required User Actions} & \textbf{User Mental Load}\\
\hline
Use Case 1                          & 15 & 3 \\
Use Case 2                          & 20 & 4 \\
Use Case 3                          & 25 & 2
\end{tabular*}

\subsection{Visual Paradigm}

% Use Case 1 - Open File
% * 1)  Indicate that you want to open a file
% * 2)  Indicate which file you want to open
% * 3)  Open the file

% Use Case 2 - Create Element at specific position
% * 1)  Indicate that you want to create an element
% * 2)  Indicate which element you want to create
% * 3)  Designate the position at which you want to create the element
% * 4)  Create the element

% Use Case 3 - Create two different elements and link them
% * 1)  Indicate that you want to create an element
% * 2)  Indicate which element you want to create
% * 3)  Designate the position at which you want to create the element
% * 4)  Create the element
% * 5)  Indicate that you want to create another element
% * 6)  Indicate which element you want to create
% * 7)  Designate the position at which you want to create the element
% * 8)  Create the element
% * 9)  Designate the first element to be linked
% * 10) Designate the second element to be linked
% * 11) Create the link

\begin{tabular*}{\textwidth}{lcc}
\textbf{Use Case} & \textbf{Required User Actions} & \textbf{User Mental Load}\\
\hline
Use Case 1                          & 15 & 3 \\
Use Case 2                          & 20 & 4 \\
Use Case 3                          & 25 & 2
\end{tabular*}

\subsection{Visual Use Case}

% Use Case 1 - Open File
% * 1)  Indicate that you want to open a file
% * 2)  Indicate which file you want to open
% * 3)  Open the file

% Use Case 2 - Create Element at specific position
% * 1)  Indicate that you want to create an element
% * 2)  Indicate which element you want to create
% * 3)  Designate the position at which you want to create the element
% * 4)  Create the element

% Use Case 3 - Create two different elements and link them
% * 1)  Indicate that you want to create an element
% * 2)  Indicate which element you want to create
% * 3)  Designate the position at which you want to create the element
% * 4)  Create the element
% * 5)  Indicate that you want to create another element
% * 6)  Indicate which element you want to create
% * 7)  Designate the position at which you want to create the element
% * 8)  Create the element
% * 9)  Designate the first element to be linked
% * 10) Designate the second element to be linked
% * 11) Create the link

\begin{tabular*}{\textwidth}{lcc}
\textbf{Use Case} & \textbf{Required User Actions} & \textbf{User Mental Load}\\
\hline
Use Case 1                          & 15 & 3 \\
Use Case 2                          & 20 & 4 \\
Use Case 3                          & 25 & 2
\end{tabular*}

\subsection{WebRatio}

% Use Case 1 - Open File
% * 1)  Indicate that you want to open a file
% * 2)  Indicate which file you want to open
% * 3)  Open the file

% Use Case 2 - Create Element at specific position
% * 1)  Indicate that you want to create an element
% * 2)  Indicate which element you want to create
% * 3)  Designate the position at which you want to create the element
% * 4)  Create the element

% Use Case 3 - Create two different elements and link them
% * 1)  Indicate that you want to create an element
% * 2)  Indicate which element you want to create
% * 3)  Designate the position at which you want to create the element
% * 4)  Create the element
% * 5)  Indicate that you want to create another element
% * 6)  Indicate which element you want to create
% * 7)  Designate the position at which you want to create the element
% * 8)  Create the element
% * 9)  Designate the first element to be linked
% * 10) Designate the second element to be linked
% * 11) Create the link

\begin{tabular*}{\textwidth}{lcc}
\textbf{Use Case} & \textbf{Required User Actions} & \textbf{User Mental Load}\\
\hline
Use Case 1                          & 15 & 3 \\
Use Case 2                          & 20 & 4 \\
Use Case 3                          & 25 & 2
\end{tabular*}

\clearpage
\appendix

\section{Essential Use Cases}

%{{{ Open File EUC ============================================================
\subsection{Open File}
\label{app:euc_open}

Opening a file is a simple, fundamental Essential Use Case for any
\ac{ide}, visual or otherwise. It refers to fetching a single file from
storage and loading it into the active editing workspace.

\begin{enumerate}
  \item Indicate that you want to open a file.
  \item Indicate which file you want to open.
  \item Open the file.
\end{enumerate}
%}}}

%{{{ Create Element EUC =======================================================
\subsection{Create Element}
\label{app:euc_create}

Visual languages are primarly composed of networks of linked elements.
Therefore, creating one such element is a fundamental Essential Use Case.
This Use Case requires that the element must be created and placed at a
specific, user defined location.

\begin{enumerate}
  \item Indicate that you want to create an element.
  \item Indicate which element you want to create.
  \item Designate the position at which you want to create the element.
  \item Create the element.
\end{enumerate}
%}}}

%{{{ Create and Link Elements EUC =============================================
\subsection{Create and Link Elements}
\label{app:euc_create_link}

Visual Languages also allow elements to share links to one another. This
Use Case requires two unique elements to be created and placed by the user,
and then linked together in a meaningful fashion.

\begin{enumerate}
  \item Indicate that you want to create an element.
  \item Indicate which element you want to create.
  \item Designate the position at which you want to create the element.
  \item Create the element.
  \item Indicate that you want to create another element.
  \item Indicate which element you want to create.
  \item Designate the position at which you want to create the element.
  \item Create the element.
  \item Designate the first element to be linked.
  \item Designate the second element to be linked.
  \item Create the link.
\end{enumerate}
%}}}

%{{{ Print an Integer EUC =====================================================
\subsection{Print an Integer}
\label{app:euc_print_integer}

\begin{enumerate}
  \item Indicate which color will be the starting color.
  \item Create the color block of size x.
  \item Determine which color represents "push".
  \item Indicate that you will begin to use that color.
  \item Create its color block.
  \item Determine which color represents "out(integer)".
  \item Indicate that you will begin to use that color.
  \item Create its color block.
  \item Determine where any black codels will need to be.
  \item Select the color black.
  \item Create the black codels.
  \item Execute the program.
\end{enumerate}
%}}}

%{{{ Create Element and Transform EUC =========================================
\subsection{Create Element and Transform}
\label{app:euc_create_transform}

\begin{enumerate}
  \item Foobar. % TODO
\end{enumerate}
%}}}

%{{{ Print a String EUC =======================================================
\subsection{Print a String}
\label{app:euc_print_string}

\begin{enumerate}
  \item 1)  Indicate that you want to create an element.
  \item 2)  Indicate which element you want to create.
  \item 3)  Designate the position at which you want to create the element.
  \item 4)  Create the element.
  \item 5)  Indicate that you want to edit the element.
  \item 6)  Select a vertex to transform.
  \item 7)  Perform transformation.
  \item 8)  Save changes to element.
\end{enumerate}
%}}}


\clearpage
\bibliography{bibliography}
\bibliographystyle{plainnat}

\end{document}
