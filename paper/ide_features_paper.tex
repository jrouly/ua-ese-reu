\documentclass[10pt,preprint]{sigplanconf}

%\usepackage[T1]{fontenc}
\usepackage[bookmarks,bookmarksopen,bookmarksdepth=2]{hyperref} %sections as bookmarks in adobe pdf
%\usepackage{times}

\newcommand{\todo}[1]{\textbf{\textcolor{red}{TODO: #1}}}
% Macros for proof-reading & corrections
\usepackage[normalem]{ulem} % for \sout
\usepackage{xcolor}
\newcommand{\ra}{$\rightarrow$}
\newcommand{\ugh}[1]{\textcolor{red}{\uwave{#1}}} % please rephrase
\newcommand{\ins}[1]{\textcolor{blue}{\uline{#1}}} % please insert
\newcommand{\del}[1]{\textcolor{red}{\sout{#1}}} % please delete
\newcommand{\chg}[2]{\textcolor{red}{\sout{#1}}{\ra}\textcolor{blue}{\uline{#2}}} % please change

% Put edit comments in a really ugly standout display
\usepackage{ifthen}
\usepackage{amssymb}
\newboolean{showcomments}
\setboolean{showcomments}{true} % toggle to show or hide comments
\ifthenelse{\boolean{showcomments}}
  {\newcommand{\nb}[2]{
    \fcolorbox{gray}{yellow}{\bfseries\sffamily\scriptsize#1}
    {$\blacktriangleright$#2$\blacktriangleleft$}
   }
   \newcommand{\version}{\emph{\scriptsize$-$working$-$}}
  }
  {\newcommand{\nb}[2]{}
   \newcommand{\version}{}
  }

\newcommand\es[1]{\nb{ES}{\textcolor{red}{\textsl{#1}}}}
\newcommand\jmr[1]{\nb{JMR}{\textcolor{red}{\textsl{#1}}}}
\newcommand\jo[1]{\nb{JO}{\textcolor{red}{\textsl{#1}}}}

% For URLs in the references
\usepackage{url}
\usepackage{hyperref}
%\usepackage{verbatim}

% Advanced Math
%\usepackage{amsmath}
%\usepackage{amssymb}     % double line font letters (for number sets N,Z,D,Q,R)

% Images and Floats
%\usepackage{epsfig}
%\usepackage{epstopdf}

%\usepackage{subfigure}
\usepackage{lscape}     % landscape
%\usepackage{array}     % align vertically in tables
\usepackage{multirow}
%\usepackage{rotating}

% Theorems, Definitions, ... (always after 'amsmath')
%\usepackage{amsthm}
%\theoremstyle{definition}
%\newtheorem{prop}{Property}%[section]
%\newtheorem{defn}{Definition}%[section]

% Algorithm environment
%\usepackage{algorithm}
%\usepackage{algorithmic}

% Commutative diagrams
%\usepackage{diagrams}

% Misc packages
%\usepackage{enumitem}
\usepackage{acronym}  % for defining acronyms


%%%%%%%%%%%%%%%%%%%%%%%%%%%%%%%%%%%%%%%%%%%%%

% Standard shortcuts
\newcommand{\eg}{\emph{e.g.,~}}             % exempli gratia (for the sake of example)
\newcommand{\ie}{\emph{i.e.,~}}             % id est (that is)
\newcommand{\etal}{~\emph{et al.}}          % et alia (and others)
\newcommand{\Fig}[1]{Fig.~\ref{#1}}         % choose Fig. or Figure, depending on the style
\newcommand{\Table}[1]{Table~\ref{#1}}      % Table reference
\newcommand{\Sect}[1]{Section~\ref{#1}}     % section name always with a capital S
\newcommand{\Model}[1]{\textsf{\small{#1}}} % name of any modeling artifact (e.g., formalism, model element, rule, ...)
\newcommand{\Code}[1]{\texttt{\small{#1}}}  % inline code
\providecommand{\e}[1]{\ensuremath{\times 10^{#1}}} % scientific notation: x.10^y

%%%%%%%%%%%%%%%%%%%%%%%%%%%%%%%%%%%%%%%%%%%%%

% Shortcuts
\newcommand{\atompm}{AToMPM}
\acrodef{ide}[IDE]{Integrated Development Environment}
\acrodef{emf}[EMF]{Eclipse Modeling Framework}
\acrodef{grc}[GRC]{GNU Radio Companion}
\acrodef{hit}[HIT]{Human Intelligence Task}
\acrodef{mt}[MTurk]{Mechanical Turk}
\acrodef{irr}[IRR]{Inter-Rater Reliability}
\acrodef{icc}[ICC]{Intra-Class Correlation}
\acrodef{uml}[UML]{Unified Modeling Language}


\begin{document}

\special{papersize=8.5in,11in}
\setlength{\pdfpageheight}{\paperheight}
\setlength{\pdfpagewidth}{\paperwidth}

\conferenceinfo{PLATEAU '14}{October 21, 2014, Portland, OR, U.S.A.} 
\copyrightyear{2014} 
\copyrightdata{978-1-nnnn-nnnn-n/yy/mm} 
\doi{nnnnnnn.nnnnnnn}

% Uncomment one of the following two, if you are not going for the 
% traditional copyright transfer agreement.

%\exclusivelicense                % ACM gets exclusive license to publish, 
                                  % you retain copyright

%\permissiontopublish             % ACM gets nonexclusive license to publish
                                  % (paid open-access papers, 
                                  % short abstracts)

%\titlebanner{banner above paper title}        % These are ignored unless
%\preprintfooter{short description of paper}   % 'preprint' option specified.

\title{Usability and Suitability Survey of Features in Visual IDEs for Non-Programmers}
%\subtitle{Subtitle Text, if any}

\authorinfo{Jean Michel Rouly}
           {George Mason University}
           {jrouly@masonlive.gmu.edu}
\authorinfo{Jonathan D. Orbeck}
           {University of Alabama}
           {jdorbeck@crimson.ua.edu}
\authorinfo{Eugene Syriani}
           {University of Montreal}
           {syriani@iro.umontreal.ca}

\maketitle

\begin{abstract}
This is the text of the abstract.
\end{abstract}

%\category{CR-number}{subcategory}{third-level}

% general terms are not compulsory anymore, 
% you may leave them out
%\terms
%term1, term2

\keywords
Integrated Development Environments, Visual Languages, Domain-Specific
Languages, Usability Study

\section{Introduction} \label{sec:introduction}

Software being a ubiquitous technology, it is nowadays being used in a wide spectrum of domains: music, teaching, arts, engineering.
The proliferation of visual \acfp{ide} has helped better assist these domain users who are unfamiliar with programming.
Common requirements to \acp{ide} in general often include~\cite{habermann1986}: uniformity and consistency across the different tools it provides, an interactive user interface for performing the different tasks, ability to inspect a well-defined state of the system being developed (\eg debugging), integrate control of versions of the system in an individual or collaborative environment, and ensure a manageable development process.

Many studies have surveyed \acp{ide}, but they were mostly focused on programming~\cite{}\es{}.
In this study, we focus on \acp{ide} used in non-programming domains where the underlying language is visual (\eg diagrams), as opposed to textual (\eg source code).

%A visual language \ac{ide} is a software tool that supports
%development within a specific visual language or set of visual languages.
%Often these \acp{ide} provide the operator with information about the
%syntax of the supported language(s) or otherwise provide features that
%integrate with any constraints of the visual language. Components and
%behaviors of the \ac{ide} interface can greatly affect the overall
%usability of an \ac{ide} as well as its suitability to its supported
%language. Consequently, it is important to utilize as many techniques as
%possible to ensure both overall interface usability and suitability to the
%target language.

%A good starting point in the analysis of visual language \ac{ide}
%suitability is an understanding of the visual languages themselves. The
%interface supporting a language must ensure that the design choices made
%within the language are mirrored and supported within the \ac{ide}.
%\cite{hils1992} For example, languages with a high level of liveness
%\cite{hils1992} would not be well suited to \acp{ide} developed with an
%indirect trigger output generation style as any benefits from liveness are
%lost. Additionally, domain-specific features of domain-customized languages
%may introduce \ac{ide} requirements to maintain full support.


%\section{Related Work} \label{sec:relatedwork}
%
%\begin{verbatim}
%TODO: Hopefully Syriani can help with this.
%\end{verbatim}
%



\section{Methods} \label{sec:methods}

The work began with the selection of a variety of visual language IDEs from
differing domains of use. This was followed by the creation and definition
of a set of novel visual IDE interface features. Finally, each of the
selected IDEs were evaluated based on the set of interface features.
\Sect{subsec:ideselection} describes the process and criteria for IDE
selection as well as the specific target IDEs selected for study.
\Sect{subsec:featuredefinition} details the set of novel interface features
devised to study the selected visual IDEs, and \Sect{subsec:ideevaluation}
describes the process of evaluation taken for each IDE.


\subsection{IDE Selection} \label{subsec:ideselection}

IDE selection was a non-scientific process with three primary criteria
driving selection. First, IDEs were selected on the basis of their support
for a visual language as the primary language of development. All IDEs
studied therefore are primarily visual language IDEs. Subsidiary criteria
for selection included popularity of the IDE within its domain as well as
representation from a variety of development domains. One goal of this
study was the evaluation of visual IDEs from a number of different domains,
so IDEs from eight independent domains were selected. The specific domains
under consideration included 3D modeling, animation, modeling, music,
prototyping, simulation, software, and workflow.


\subsection{Feature Definition} \label{subsec:featuredefinition}

Through observation and use of the IDEs, we began noting the differences
and similarities present that defined either interface purpose or
functionality. We determined whether these interface features could be
applied to all IDEs and subsequently formalized the definitions of all
universally applicable features. Some variables were strongly influenced by
existing literature. All variables were categorized under five headings:
Audience, Chrome, Human Interface, Integration, and Language Syntax.
\Sect{subsubsec:audience} --- \Sect{subsubsec:languagesyntax} detail these
categories and constituent features.


\subsubsection{Audience Features} \label{subsubsec:audience}

The audience of an IDE refers generally to the target populations for which
the IDE is developed or intended. Features under the audience category
attempt to describe those meta-features of IDEs.


\paragraph{Domain} The domain of an IDE refers to the domain or field of
knowledge under which the interface falls. That is, the field or fields for
which the interface is primarily used, if any. This is a nominal variable.
Example values might include \texttt{General}, \texttt{Modeling},
\texttt{Software}. A value of \texttt{General} indicates that the IDE is
general purpose and can be applied to or used within any number of
different fields to the same extent.


\paragraph{Skill Level} The skill level of an IDE describes what level of
skill in the IDE's domain is expected of users. This is a nominal variable.
Possible values are \texttt{Novice}, \texttt{Intermediate},
\texttt{Expert}, and \texttt{General} with a value of \texttt{General}
indicating that the IDE offers a powerful set of advanced features while
maintaining components that emphasize accessibility and ease of use.


\subsubsection{Chrome Features} \label{subsubsec:chrome}

The chrome of an IDE is the total set of all user interface components
external to the workspace. This includes every tool, menu, button, or other
user interface component not contained within the workspace area.


\paragraph{Popular Features} Many IDEs provide support for a specific
subset of common operations. \citeauthor{murphy2006} defines a list of the
top ten IDE features most frequently utilized by developers which is
reproduced here as a set of boolean sub-variables. Each sub-variable
indicates whether the IDE in question supports the use of that particular
features. This is a compound variable.

\subparagraph{Delete} Delete a syntactic element in the workspace.

\subparagraph{Save} Save and export a model to storage media.

\subparagraph{Paste} Duplicate an existing syntactic element in the
work{-}space from the paste buffer.

\subparagraph{Content Assist} Provide suggestions or completion for
elements.

\subparagraph{Copy} Place an existing syntactic element from the
work{-}space into paste buffer.

\subparagraph{Undo} Undo the user's most recent action.

\subparagraph{Cut} Remove a syntactic element from the workspace and place
it in the paste buffer.

\subparagraph{Refresh} Load contents of workspace and interface dashboard
elements from storage media and update display if necessary.

\subparagraph{Show View} Open and display a new tool in the interface.

\subparagraph{Next Word} Move active selection to the next element
according to some natural ordering.


\paragraph{Context Sensitive Tools} Any interface component which changes
visibly or is generated anew depending on the context of selected elements
within the workspace is context sensitive. This is boolean variable
indicates whether context sensitive tools are supported.


\paragraph{Degree of Interface Visual Richness} The degree of an IDE's
interface visual richness describes the extent to which it utilizes visual
variables to increase the visual discriminability of available tools. This
is a compound variable composed of eight sub-variables. Each sub-variable
is a boolean measure determining whether or not the described visual
variable is utilized in the interface to distinguish available tools.

\subparagraph{Icons} Images contained in a border of a standard size and
shape representing distinct actions or tools.
\cite{costagliola2002,moody2009}

\subparagraph{Shape} Distinct shapes indicate different tools.
\cite{moody2009}

\subparagraph{Size} Different tool sizes indicate distinct tools.
\cite{moody2009}

\subparagraph{Color} Color is used to indicate distinct tools.
\cite{moody2009}

\subparagraph{Text} Text (or typographic variation) is used to identify or
distinguish tools. \cite{moody2009}

\subparagraph{Organizational Coherence} Components with related purpose are
visually grouped in the interface. \cite{constantine1996}

\subparagraph{Texture} Shading or shadows are used to modify tools or to
distinguish between distinct tools. \cite{moody2009}

\subparagraph{Brightness} The brightness of a color(\ie its perceived
luminosity) is used to indicate a difference between tools.
\cite{moody2009}


\paragraph{Multiplicity of Perspectives} This ratio variable encapsulates
the number of available predefined interface perspectives available to the
user. A perspective is defined as a visual configuration of the available
tools in the interface chrome and elements in the workspace for the
purposes of accomplishing a distinct task by mean of a distinct process.
Values can range anywhere greater than zero.


\paragraph{Object Properties Window} The object properties window is any
interface component that displays the properties of an element in the
workspace. This component generally also allows modification of element
properties. This is a nominal variable. If no object properties window is
available, this variable takes the value \texttt{None}. Otherwise,
available values are \texttt{Omnipresent} and \texttt{Manual}. The former
refers to a window which is always present, allowing contents to update
contextually. The latter refers to a window which requires user interaction
to bring forward.


\paragraph{Searchable Toolspace} This is a boolean variable which indicates
whether the total set of available tools, components, or actions offered by
the IDE can be searched through by name or keyword.


\paragraph{Toolbar Styles} The toolbar styles of an IDE refer to the set of
user interface component idioms employed by the IDE. \cite{galitz2007} This
nominal variable can take combinations of multiple values. Example values
might include \texttt{Icons}, \texttt{Menus}, \texttt{Ribbons},
\texttt{Trees}.


\paragraph{Visual Clutter} The clutter of an interface is defined as the
number and organization of tools available on the screen versus the amount
of workspace provided by the IDE. If the IDE offers no method for tool
organization or if there is an immense amount of tools visible at once,
then the IDE is likely to be visually cluttered. Visual clutter is a
nominal variable and can take the values \texttt{Low}, \texttt{Medium}, or
\texttt{High}. \Sect{subsec:ideevaluation} includes details on the proper
evaluation of this qualitative feature.

\subsubsection{Human Interface Features} \label{subsubsec:humaninterface}
\subsubsection{Integration Features} \label{subsubsec:integration}
\subsubsection{Language Syntax Features} \label{subsubsec:languagesyntax}


\subsection{IDE Evaluation} \label{subsec:ideevaluation}


\section{Results}
\label{sec:results}


\paragraph{Alice3} Designed primarily as an educational programming
environment, Alice3 alienates skilled users by expecting a lower level of
skill in its domain. It makes up for this loss in accessibility, however,
with its wide, almost universal, support of common \ac{ide} features as
well as context-sensitive tooling. Alice3 provides a medium level of visual
richness in its interface chrome, but boasts one of the highest essential
efficiency values. This efficiency value is achieved through the ``MIT
Scratch'' style of visual syntax. However, despite greatly reducing
operator mental load, Alice3's clunky design only manages a near-neutral
interface efficiency. The introduction of optional keybindings is a
redeeming factor for Alice3 and, like most educational programming
\acp{ide}, Alice3 includes simple modular complexity management and a
visually appealing level of language visual richness.

%% TODO
\paragraph{AToMPM}

\paragraph{AudioMulch} Although AudioMulch offers a wide array of tools and
an in-depth interface ideal for professionals in the music industry, the
overall complexity of the design greatly reduces the accessibility for
anyone else. It supports a large amount of popular \ac{ide} features,
however, and offers equally high essential and interface efficiency
ratings. This is visually assisted by a relation-highlighting feature
which also provides AudioMulch with an implicit syntax enforcement, both of
which greatly aid the user in model creation. Unfortunately, there is no
effort made to manage the high amount of complexity within the \ac{ide} and
virtually all of the canvas elements look exactly the same, ultimately
awarding AudioMulch with a low language visual richness score.

\paragraph{Blender} Designed for a skilled target audience of experts in
the domain, it is no surprise that Blender supports most common \ac{ide}
features or provides context-sensitive tooling. Its high level of chrome
visual richness and the large number of perspectives available relative to
the average found in this study also contribute to a high quality
interface. However, its featurefulness leads directly to the second highest
observed value for visual clutter. Blender possess no particular efficiency
techniques, remaining around a perfect one-to-one relationship with the
measured essential use cases. Its heavy use of the keyboard reduces
accessibility to a wider audience, although the target skill level is
already a limiting factor. Finally, Blender provides modularization
complexity management through saving and duplication tools,along with one
of the most visually rich languages observed.

\paragraph{Cameleon} Intended for rapid, flexible visual prototyping of
functional algorithms, Cameleon is relatively feature impoverished --- it
only supports three of the most popular \ac{ide} features. It does however
boast a large number of available tools which are conveniently searchable.
Cameleon's interface chrome employs five visual richness variables,
slightly above the average count. Its sleek, simple interface is rated as
low clutter, and provides a positive level of essential and interface
efficiency. Interaction requires use of the keyboard for non-essential
actions, specifically advanced navigation and zooming. Cameleon supports
the user by interactively highlighting allowed syntactic relations and
explicitly enforcing syntax requirements. More support is provided by its
hierarchical complexity management system. Overall, Cameleon is an
efficient, simple tool with a large library of functionality and a focus on
supporting syntax requirements.

\paragraph{\acl{emf}} The \ac{emf} is one of the most sophisticated
\acp{ide} in this study and demands an expert level of skill from target
users. It is also the only \ac{ide} studied which supports all of the
measured popular features, as any expert in the domain would likely come to
expect. The \ac{emf} interface chrome supports five visual richness
variables as well as a large number of predefined interface perspectives.
The high level of supported perspectives, while perhaps intimidating,
increases the overall power and utility of the interface. Additionally, the
\ac{emf} tool space is searchable, which counterbalances the highest
observed level of visual clutter. Despite being extremely cluttered,
\ac{emf} has average levels of essential efficiency and the highest
measured level of interface efficiency. The supported visual languages do
not make use of more than five visual variables but do allow for complexity
management. Overall, \ac{emf} is well suited for an expert user with high
efficiency and several convenience features, but is too cluttered and not
visually rich enough to support a wider audience of varied skill levels.

\paragraph{\acl{grc}} \acl{grc} is a simple platform designed to aid the
development of signal processing software without the need to understand or
write code. It does, however, require a moderate level of skill in the
domain. \ac{grc} supports slightly more than the average number of popular
\ac{ide} features as well as the ability to search through available tools.
\ac{grc} is not extensively visually pleasing, however, as the interface
chrome only utilizes four visual variables while the supported language
only employs two. Despite this, the interface is highly efficient, with the
highest observed interface efficiency and a correspondingly high level of
essential efficiency. It also supports some optional keyboard use and
explicit syntax checking. \ac{grc} is designed as an easy to use interface
for non-coders, and manages to maintain simplicity while still offering a
large amount of technical power.

\paragraph{Grasshopper 3D} Though Grasshopper is able to provide the user
with a relatively simple and easy to use interface, beginners would likely
shy away from the complexity of its core functionality. Even so, it offers
a high amount of popular \ac{ide} features and context sensitive tools, as
well as the ability to search through its vast library of tools quickly and
easily by name. Grasshopper also possesses very good efficiency techniques,
maintaining both values at a more-than-decent level. In addition, the
optional use of a keyboard is supported, offering functionality on another
level to increase accessibility. Complexity management is not supported at
all, however, and visual richness in both the language and tools are
mediocre at best.

%% TODO
\paragraph{Max}

\paragraph{MetaEdit+} Aimed towards an intermediate level of users, the
many different tasks and steps that go into the design of a simple model in
MetaEdit+ can easily be daunting for newer users.  These tasks are
fortunately divided through modularization, allowing them to be much more
manageable withing the \ac{ide}. Furthermore, many of the popular \ac{ide}
features as well as context sensitive tools are present in the interface,
increasing the accessability even more. MetaEdit+ also holds the lowest
clutter value and integrates an inplicit syntax enforcement, providing the
user with a clear and easy to use workspace. A favorable essential
efficiency value is also present, whereas the interface efficiency suffers
from MetaEdit+'s necessity to complete a dialog for each created element.

\paragraph{MIT AppInventor} Much like Alice3 and the other educational
interfaces, MIT's AppInventor is designed for an unskilled, novice
audience, reducing the breadth of target audience. By only supporting three
of the top ten popular features, AppInventor additionally alienates its
audience through nonconformity with expected standards. While colorful,
AppInventor's interface only supports four visual variables to distinguish
elements in the chrome, relying heavily on icons and text. AppInventor
combines the ``MIT Scratch'' style of visual syntax along with a Drag n
Drop element creation workspace, resulting in high levels of effective and
interface efficiency. Another artifact of the ``MIT Scratch'' style of
visual syntax is its implicit syntax enforcement, assisting the user by
preventing illegal structures. Finally, the high level of language visual
richness and modular complexity management scheme result in an overall
visually pleasant experience.

\paragraph{MST Workshop} Though MST Workshop offers a very simple and
easy to use interface, it does not offer much explanation as to the use of
its many different simulation categories. As such, they are virtually
unusable by anyone without prior knowledge of that subject, drastically
limiting its accessibility. In the same vein, it does not incorporate
enough visual richness variables to easily discern or interpret the tools
and very few popular \ac{ide} features are supported. Though the efficiency
values are better than decent and the clutter was rated to be relatively
low, no actions were made to reduce the complexity of the system or even
enforce the language's syntax. All of this merged together with a
less-than-stellar language visual richness level shows that MST Workshop
definitely has room for improvement.

\paragraph{Piet Creator} As the primary \ac{ide} used in the creation of
Piet programs, Piet Creator does a very good job of providing a very simple
interface for novices without limiting the usability for more expert users.
On top of that, the toolbars utilize seven of the eight tool visual
richness variables and it possesses an extremely low clutter rating,
maximizing accessibility for all users.  However, Piet Creator does not
offer any sort of properties dialog, eliminating the ability to manage data
on a deeper level. The user is also limited to using only the mouse for
every task, which creates a poor combination with the fact that Piet
Creator holds the lowest interface efficiency value. The Piet language
syntax is not enforced in the slightest, forcing the user to manually debug
his/her whole program to located any errors in the code. In general, the
simplicity of Piet Creator's design allows for a relaxed and visually
appealing environment, but it can evidently be overly threadbare to a
degrading extent.

\paragraph{Scratch} One of the earliest educational interfaces present in
this study, Scratch has influenced a great deal of later \acp{ide},
including Stencyl, AppInventor, and Alice3 to name a few. Scratch is
specifically targeted toward a novice audience with a low skill level, and
only supports three of the top ten popular features. The interface employs
an average level of visual richness variables, and does not even define an
object properties window. The sparse, cluttered interface redeems itself
through high values of essential and interface efficiency. Implicit syntax
enforcement provides a safe environment for learning users, and the highest
observed level of language visual richness provides an engaging, visually
rich display. The focus on efficiency and visual richness works well in an
\ac{ide} designed for education by accelerating reinforcement and engaging
student attention.

\paragraph{Simulink} Though Simulink features a plethora of components that
can allow it to perform virtually any electrical simulation, the vast scope
of its functions and the amount of on-screen tasks greatly reduces its
accessibility. Each individual function that Simulink provides creates its
own dialog window on screen, severely increasing visual clutter and
complexity with prolonged use and no techniques for complexity management.
A searchable toolspace is present to take off some mental load in dealing
with Simulink's huge tool libraries, however very few visual richness
variables are integrated to increase discernibility between the tools. The
canvas elements also provide some relatively good efficiency values, though
the \ac{ide} provides no features to enforce the syntax of the simulation
language. The lack of visual richness overall detracts from the usability
and enjoyment of this tool, despite it being very powerful.

\paragraph{Stencyl} Stencyl provides a very easy to use interface for game
software creation for an early level audience, but in the process
sacrifices any higher level functionality. This \ac{ide} also integrates
the ``MIT Scratch'' style of coding, which carries along with it very high
efficiency values and implicit syntax enforcement. Every tool visual
richness variable is utilized within the toolbars, which are also
searchable, allowing for a very user-friendly and productive interface.
Each language visual richness variable is supported within the canvas as
well, thereby giving Stencyl perfect visual richness ratings. A very large
number of perspectives are present due to Stencyl's powerful modularization
techniques, greatly reducing the amount of mental strain on the user to
handle the many steps that go into creating a game and showing Stencyl to
be an altogether organized and well-designed \ac{ide}.

\paragraph{Tersus} Tersus is able offer an interface for web application
design with a very large number of popular \ac{ide} features, largely
thanks to the integration of Eclipse's user interface. Unfortunately, very
few visual richness variables are utilized for either the tools or the
language and there is no syntax enforcement to aid the user at all through
the design process. The efficiency of the \ac{ide} is decent, however, and
Tersus hits a high point with its use of a hierarchical design to program
behaviors and functions of the various components in the model. Though this
is a very nice feature, it doesn't distract at all from the overall
blandness of the language.

\paragraph{TouchDevelop}

\paragraph{UMLet} UMLet provides a general purpose, easy to use software
interface with no particular expectations about user skill level. It offers
many of the most popular \ac{ide} features a user might expect, but does
not offer a visually rich chrome or language. Conceptually efficient but
clunky in implementation, UMLet measured high in essential efficiency but
low in interface efficiency. To the detriment of the inexperienced user,
UMLet offers no syntax enforcement. It also offers no complexity management
devices to ease mental load. Barebones to the extreme, UMLet provides users
with a visually simple interface and no additional usability features like
context-sensitive tooling or searchable tools. Its power, however, lies in
its simplicity and ease of use at any entry skill level.

\paragraph{Violet} Similar to UMLet, Violet provides an easy to use
interface with no specific entry skill level assumptions. It offers the
same common \ac{ide} features as UMLet, with a slightly richer chrome.
Violet's interface is slightly less cluttered than UMLet's but also offers
a lower essential efficiency. Violet is higher in interface efficiency over
UMLet, though. Neither interface offers syntax enforcement or complexity
management and, since both support UML as the target visual language, both
share a low level of language visual richness. The two interfaces are
relatively featureless and differ primarily in the mode of user
interaction.

%% TODO
\paragraph{VisSim}

%% TODO
\paragraph{Visual Paradigm}

%% TODO
\paragraph{Visual Use Case}

%% TODO
\paragraph{WebRatio}

\paragraph{YAWL} YAWL is a workflow system that supports a simple, UML-like
modeling language. Despite its simplicity, YAWL is targeted toward skilled
members of its domain. YAWL is lacking in its support of popular \ac{ide}
features and any form of active syntax checking. Complex diagrams are also
difficult to manipulate given the lack of complexity management paradigms.
YAWL also only supports a slightly lower than average number of visual
richness variables in both the chrome and language. The language itself
especially suffers from lack of visual richness with highly visually
similar, square elements. However, YAWL's interface has a very low clutter
value and high degrees of efficiency. Ultimately, the negatives outweigh
the positives and, though minimal and efficient, YAWL is overall a visually
boring, feature impoverished \ac{ide}.


\section{Discussion}
\label{sec:discussion}

The feature that exhibited the most agreement between \acp{ide} was the
multiplicity of available perspectives, with 13 \acp{ide} containing only
single available perspective. The next most agreed upon features are
interface visual richness and language visual richness which both indicate
nine \acp{ide} that support only four visual richness variables for either
interface or language. Note that \acp{ide} do not necessarily employ the
same number of visual richness variables for interface as well as supported
language.


\begin{verbatim}
+ what features were most influential
    (toward usability / suitability)
+ what features were least
+ any IDEs that particularly stood
    out in the process
\end{verbatim}


\section{Conclusion}
\label{sec:conclusion}

The usability and suitability of an \ac{ide} can begin to be understood
through an analysis of its interface characteristics. Design decisions
involving the intended audience, the chrome or interface of the \ac{ide},
the style of human interaction, and features that affect level of language
support all impact the overall usability and suitability. Even simple
convenience features, like context sensitivity of searchable tools, can
positively impact usability. Meanwhile, a lack of more integral characteristics,
like visual richness, can very negatively impact the quality of an \ac{ide}.

This is a preliminary study that is not exhaustive. Additional visual \acp{ide} need to be
considered to grow the body of collected data. Research into new features,
revision of existing features, and additional analytical steps post data
collection would further the gained understanding of studied \acp{ide}.
The goal of this study has been to provide a technical foundation for the
systematic analysis and understanding of domain-specific visual \acp{ide}.


%\appendix
%\section{Appendix Title}

\acks

This research was sponsored by the National Science Foundation grant no. 1156563 at the University of Alabama REU site.

\bibliographystyle{abbrvnat}
\bibliography{bibliography/biblio,bibliography/biblio-reu,bibliography/ides}

\end{document}
