\section{Conclusion}
\label{sec:conclusion}

The usability and suitability of an \ac{ide} can begin to be understood
through an analysis of its interface characteristics. Design decisions
involving the intended audience, the chrome or interface of the \ac{ide},
the style of human interaction, and features that affect level of language
support all impact the overall usability and suitability. Even simple
convenience features, like context sensitivity of searchable tools, can
positively impact usability. Meanwhile, a lack of more integral characteristics,
like visual richness, can very negatively impact the quality of an \ac{ide}.

The features discussed in this paper are inspired from software engineering approaches and methodologies.
A more in-depth list of feature based on human-computer interaction theories will complement this study,
such as the work by Green and Petre~\cite{Green1996} of incorporating a cognitive dimension in visual programming environments.
This is left for future work.

This is a preliminary study that is not exhaustive. Additional visual \acp{ide} need to be
considered to grow the body of collected data. Research into new features,
revision of existing features, and additional analytical steps that follow after data
collection would further the gained understanding of studied \acp{ide}.
The goal of this study has been to provide a technical foundation for the
systematic analysis and understanding of domain-specific visual \acp{ide}.
