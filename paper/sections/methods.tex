\section{Methods} \label{sec:methods}

The work began with the selection of a variety of visual language IDEs from
differing domains of use. This was followed by the creation and definition
of a set of novel visual IDE interface features. Finally, each of the
selected IDEs were evaluated based on the set of interface features.
\Sect{subsec:ideselection} describes the process and criteria for IDE
selection as well as the specific target IDEs selected for study.
\Sect{subsec:featuredefinition} details the set of novel interface features
devised to study the selected visual IDEs, and \Sect{subsec:ideevaluation}
describes the process of evaluation taken for each IDE.


\subsection{IDE Selection} \label{subsec:ideselection}

IDE selection was a non-scientific process with three primary criteria
driving selection. First, IDEs were selected on the basis of their support
for a visual language as the primary language of development. All IDEs
studied therefore are primarily visual language IDEs. Subsidiary criteria
for selection included popularity of the IDE within its domain as well as
representation from a variety of development domains. One goal of this
study was the evaluation of visual IDEs from a number of different domains,
so IDEs from eight independent domains were selected. The specific domains
under consideration included 3D modeling, animation, modeling, music,
prototyping, simulation, software, and workflow.


\subsection{Feature Definition} \label{subsec:featuredefinition}

Through observation and use of the IDEs, we began noting the differences
and similarities present that defined either interface purpose or
functionality. We determined whether these interface features could be
applied to all IDEs and subsequently formalized the definitions of all
universally applicable features. Some variables were strongly influenced by
existing literature. All variables were categorized under five headings:
Audience, Chrome, Human Interface, Integration, and Language Syntax.
\Sect{subsubsec:audience} --- \Sect{subsubsec:languagesyntax} detail these
categories and constituent features.


\subsubsection{Audience Features} \label{subsubsec:audience}

The audience of an IDE refers generally to the target populations for which
the IDE is developed or intended. Features under the audience category
attempt to describe those meta-features of IDEs.


\paragraph{Domain} The domain of an IDE refers to the domain or field of
knowledge under which the interface falls. That is, the field or fields for
which the interface is primarily used, if any. This is a nominal variable.
Example values might include \texttt{General}, \texttt{Modeling},
\texttt{Software}. A value of \texttt{General} indicates that the IDE is
general purpose and can be applied to or used within any number of
different fields to the same extent.


\paragraph{Skill Level} The skill level of an IDE describes what level of
skill in the IDE's domain is expected of users. This is a nominal variable.
Possible values are \texttt{Novice}, \texttt{Intermediate},
\texttt{Expert}, and \texttt{General} with a value of \texttt{General}
indicating that the IDE offers a powerful set of advanced features while
maintaining components that emphasize accessibility and ease of use.


\subsubsection{Chrome Features} \label{subsubsec:chrome}

The chrome of an IDE is the total set of all user interface components
external to the workspace. This includes every tool, menu, button, or other
user interface component not contained within the workspace area.


\paragraph{Popular Features} Many IDEs provide support for a specific
subset of common operations. \citeauthor{murphy2006} defines a list of the
top ten IDE features most frequently utilized by developers which is
reproduced here as a set of boolean sub-variables. Each sub-variable
indicates whether the IDE in question supports the use of that particular
features. This is a compound variable.

\subparagraph{Delete} Delete a syntactic element in the workspace.

\subparagraph{Save} Save and export a model to storage media.

\subparagraph{Paste} Duplicate an existing syntactic element in the
work{-}space from the paste buffer.

\subparagraph{Content Assist} Provide suggestions or completion for
elements.

\subparagraph{Copy} Place an existing syntactic element from the
work{-}space into paste buffer.

\subparagraph{Undo} Undo the user's most recent action.

\subparagraph{Cut} Remove a syntactic element from the workspace and place
it in the paste buffer.

\subparagraph{Refresh} Load contents of workspace and interface dashboard
elements from storage media and update display if necessary.

\subparagraph{Show View} Open and display a new tool in the interface.

\subparagraph{Next Word} Move active selection to the next element
according to some natural ordering.


\paragraph{Context Sensitive Tools} Any interface component which changes
visibly or is generated anew depending on the context of selected elements
within the workspace is context sensitive. This is boolean variable
indicates whether context sensitive tools are supported.


\paragraph{Degree of Interface Visual Richness} The degree of an IDE's
interface visual richness describes the extent to which it utilizes visual
variables to increase the visual discriminability of available tools. This
is a compound variable composed of eight sub-variables. Each sub-variable
is a boolean measure determining whether or not the described visual
variable is utilized in the interface to distinguish available tools.

\subparagraph{Icons} Images contained in a border of a standard size and
shape representing distinct actions or tools.
\cite{costagliola2002,moody2009}

\subparagraph{Shape} Distinct shapes indicate different tools.
\cite{moody2009}

\subparagraph{Size} Different tool sizes indicate distinct tools.
\cite{moody2009}

\subparagraph{Color} Color is used to indicate distinct tools.
\cite{moody2009}

\subparagraph{Text} Text (or typographic variation) is used to identify or
distinguish tools. \cite{moody2009}

\subparagraph{Organizational Coherence} Components with related purpose are
visually grouped in the interface. \cite{constantine1996}

\subparagraph{Texture} Shading or shadows are used to modify tools or to
distinguish between distinct tools. \cite{moody2009}

\subparagraph{Brightness} The brightness of a color(\ie its perceived
luminosity) is used to indicate a difference between tools.
\cite{moody2009}


\paragraph{Multiplicity of Perspectives} This ratio variable encapsulates
the number of available predefined interface perspectives available to the
user. A perspective is defined as a visual configuration of the available
tools in the interface chrome and elements in the workspace for the
purposes of accomplishing a distinct task by mean of a distinct process.
Values can range anywhere greater than zero.


\paragraph{Object Properties Window} The object properties window is any
interface component that displays the properties of an element in the
workspace. This component generally also allows modification of element
properties. This is a nominal variable. If no object properties window is
available, this variable takes the value \texttt{None}. Otherwise,
available values are \texttt{Omnipresent} and \texttt{Manual}. The former
refers to a window which is always present, allowing contents to update
contextually. The latter refers to a window which requires user interaction
to bring forward.


\paragraph{Searchable Toolspace} This is a boolean variable which indicates
whether the total set of available tools, components, or actions offered by
the IDE can be searched through by name or keyword.


\paragraph{Toolbar Styles} The toolbar styles of an IDE refer to the set of
user interface component idioms employed by the IDE. \cite{galitz2007} This
nominal variable can take combinations of multiple values. Example values
might include \texttt{Icons}, \texttt{Menus}, \texttt{Ribbons},
\texttt{Trees}.


\paragraph{Visual Clutter} The clutter of an interface is defined as the
number and organization of tools available on the screen versus the amount
of workspace provided by the IDE. If the IDE offers no method for tool
organization or if there is an immense amount of tools visible at once,
then the IDE is likely to be visually cluttered. Visual clutter is a
nominal variable and can take the values \texttt{Low}, \texttt{Medium}, or
\texttt{High}. \Sect{subsec:ideevaluation} includes details on the proper
evaluation of this qualitative feature.

\subsubsection{Human Interface Features} \label{subsubsec:humaninterface}
\subsubsection{Integration Features} \label{subsubsec:integration}
\subsubsection{Language Syntax Features} \label{subsubsec:languagesyntax}


\subsection{IDE Evaluation} \label{subsec:ideevaluation}
