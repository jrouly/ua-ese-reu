\section{Methods}
\label{sec:methods}

\begin{verbatim}
*Find IDEs to study*

  -investigated popular IDEs that utilized a syntactically visual language
  -IDEs were chosen over a variety of domains for a thourough, overarching comparison

*Where we got variables from / process for their creation*

  -Through observation of the various IDEs, we found the differences and commonalities present within the interfaces that defined either the purpose or the functionality of the IDE
  -Other variables influenced by other writings (popular features, essential efficiency, visual richness)
  -These variables were catagories under either audience, chrome, human interface, integration, and language syntax (add definitions from tech report)
  -Values for the variables were determined by focusing on one variable across each of the IDEs, grouping together the IDEs that integrated similar functionality within that individual variable, and finding a way to define that functionality
  -(add discussion about each variable and thier values)

*Use of the table to store values*

  -After defining each of the variables and thier possible values, we created a table to store information about each of the IDEs as they relate to the variables
  -Extra tables created for compound variables (i.e. visual richness, popular features)

*Efficiency - how did we measure it, what were the Use Cases, etc.*

  -Brainstormed three general purpose essential use cases of increasing complexity to test within each IDE
  -Open a file, create an element, create and link two elements/output a string/output an integer/translate a point
  -Defined every step for each use case
  -Worked through each use case for every IDE
  -Made note of which steps contributed to mental load and the number of physical user actions for each use case
  -Determined the third and most complex use case to be most representative of the IDEs' efficiency
  -Final value calculated by dividing the number of mental load steps/physical user actions by the number of steps in the use case and subtracting that value by 1

*Mechanical Turk*

  -Designed a survey in which MTurk users would look at screenshots of the IDEs and rate on a scale of 1 to 5 how "cluttered" they determined the screenshot to be
  -Provided a formal definition of visual clutter as a reference and guideline for the workers to base thier decisions off of
  -Took three screenshots each of the IDEs at various levels of use to balance varying amounts of clutter within one IDE
  -Grayed out the workspace on each screenshot to prevent any model complexity from interfering with the users' rating
  -Limited each screenshot to 5 unique workers in order to prevent duplicity and collect enough usable data
  -Calculated worker average for each screenshot and sorted them within their respective IDE by this value
  -Calculated the average over the screenshots for each IDE and used this as the final value

*How did we score IDEs / how did we determine scoring metrics*
  -For each nominal or boolean variable, we used our experience with the 25 IDEs to justify which values were beneficial or somewhat beneficial to the IDE, which detracted from its use, and which did not affect it either way
  -For each interval variable, we set thresholds based on collected values 
  -The scores of thes values were weighed against eachother within each IDE to produce the survey of the suitability and usability of various visual integrated development enviornments
\end{verbatim}
