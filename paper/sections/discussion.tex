\section{Discussion}
\label{sec:discussion}

A visual depiction of the features studied is given in
\Fig{fig:featuremodel} as a feature diagram model. Observed relationships
between these features are discussed in this section.

The feature that exhibited the most agreement between \acp{ide} was the
multiplicity of available perspectives, with thirteen \acp{ide} containing
only a single available perspective. The next most agreed upon features are
interface visual richness and language visual richness which both indicate
nine \acp{ide} that support only four visual richness variables for either
interface or language. Note that \acp{ide} do not necessarily employ the
same number of visual richness variables for interface as well as supported
language.

Many \acp{ide} would be able to benefit through the implementation of
simple, positive features like the addition of visual richness variables.
Additionally, less than half of studied \acp{ide} implement context
sensitive menus (only eleven do). Even simple convenience features like the
ability to search through available tools are common in less than half of
studied \acp{ide}.

Some \ac{ide} features can be related. It is evident from the collected
data that interface and essential efficiency are significantly related. A
Pearson's product-moment correlation test indicates that the two variables
share a correlation coefficient of $0.556$ with $p=0.003893$. Neither
interface nor essential efficiency significantly correlate with visual
clutter however, with $p>0.4$ for each.
