\section{Introduction}
\label{sec:introduction}

\begin{verbatim}
Hopefully Syriani can help with this.
\end{verbatim}

\subsection{Related Work}
\label{subsec:related_work}

A visual language \ac{ide} is a software tool that supports
development within a specific visual language or set of visual languages.
Often these \acp{ide} provide the operator with information about the
syntax of the supported language(s) or otherwise provide features that
integrate with any constraints of the visual language. Components and
behaviors of the \ac{ide} interface can greatly affect the overall
usability of an \ac{ide} as well as its suitability to its supported
language. Consequently, it is important to utilize as many techniques as
possible to ensure both overall interface usability and suitability to the
target language.

\subsubsection{Language Analysis}
A good starting point in the analysis of visual language \ac{ide}
suitability is an understanding of the visual languages themselves. The
interface supporting a language must ensure that the design choices made
within the language are mirrored and supported within the \ac{ide}.
\cite{hils1992} For example, languages with a high level of liveness
\cite{hils1992} would not be well suited to \acp{ide} developed with an
indirect trigger output generation style as any benefits from liveness are
lost. Additionally, domain-specific features of domain-customized languages
may introduce \ac{ide} requirements to maintain full support.
