\section{Introduction} \label{sec:introduction}

Software being a ubiquitous technology, it is nowadays being used in a wide spectrum of domains: music, teaching, arts, engineering.
The proliferation of visual \acfp{ide} has helped better assist these domain users who are unfamiliar with programming.
Common requirements to \acp{ide} in general often include~\cite{habermann1986}: uniformity and consistency across the different tools it provides, an interactive user interface for performing the different tasks, ability to inspect a well-defined state of the system being developed (\eg debugging), integrate control of versions of the system in an individual or collaborative environment, and ensure a manageable development process.

Previous studies have surveyed \acp{ide}, but they were mostly focused on the
programming domain~\cite{hils1992, fischer1994, habermann1986}.
In this study, we focus on \acp{ide} used in non-programming domains where the underlying language is visual (\eg diagrams), as opposed to textual (\eg source code).
Often these \acp{ide} provide the user with information about the syntax of the supported language(s) or otherwise provide features that integrate with any constraints of the visual language.
Components and behaviors of the \ac{ide} interface can greatly affect the overall usability of an \ac{ide} as well as its suitability to its supported language.
Consequently, it is important to utilize as many techniques as possible to ensure both overall interface usability and suitability to the target language.
Several works have looked at metrics for measuring the usability of
software~\cite{constantine1996,Green1996}.
We have adapted these metrics to specifically look at visual \acp{ide} in different domains.

%Certain generative product line approaches, such as domain-specific modeling (DSM)~\cite{Kelly2008}, are concerned with developing languages and tools tailored to users of a specific domain.
%To favor reuse, DSM developers typically construct an \ac{ide} that can be automatically customized to handle a DSM language.
%Because of this generic approach, the resulting \acp{ide} are most of the time not optimally usable and suitable for users in different domains, such as biology, music, engineering, and so on.
The goal of this study is to evaluate existing \acp{ide} in a variety of domains to make the developers of cross-domain generic \acp{ide} aware of the different features needed to make their tools more usable and suitable by their target users.
generative product line approaches, such as domain-specific modeling (DSM)~\cite{Kelly2008}, would highly benefit from such a study.

%To bridge the cognitive gap between a specific domain and the world of software, the domain-specific modeling (DSM) approach provides tools and languages for domain-experts to design and manipulate models using the notations they are most familiar with and hide all the complexity software programming adds~\cite{Kelly2008}.
%However, much of the research in DSM has been focusing on the software application domains and not enough on completely different domains, such as biology, music, engineering, and so on.
%Therefore, the tools and languages developed are still not optimally suitable and usable for users in these domains.
%The goal of this study is to provide DSM developers a preliminary set of means to evaluate features of the \acp{ide} they generate for different domains.

\Sect{sec:methods} describes the methodology and features covered by our survey.
\Sect{sec:results} summarizes the results of the study for each \acp{ide} surveyed.
In \Sect{sec:discussion}, we discuss interpretation and limitations of the survey, and conclude in \Sect{sec:conclusion}.

%A good starting point in the analysis of visual language \ac{ide}
%suitability is an understanding of the visual languages themselves. The
%interface supporting a language must ensure that the design choices made
%within the language are mirrored and supported within the \ac{ide}.
%\cite{hils1992} For example, languages with a high level of liveness
%\cite{hils1992} would not be well suited to \acp{ide} developed with an
%indirect trigger output generation style as any benefits from liveness are
%lost. Additionally, domain-specific features of domain-customized languages
%may introduce \ac{ide} requirements to maintain full support.
