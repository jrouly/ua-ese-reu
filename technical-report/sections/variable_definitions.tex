\section{Variable Definitions}
\label{sec:definitions}

%{{{ subsection: Audience
\subsection{Audience}
\label{subsec:audience}

The \textbf{Audience} of an \ac{ide} refers generally to the target
populations for which the \ac{ide} is developed.

%{{{ Subsubsection: Domain
\subsubsection{Domain}
\label{subsubsec:domain}

\begin{AlignedDesc}
  \item[Abbreviation] \texttt{Domain}

  \item[Variable Type] Nominal

  \item[Description] The \textbf{Domain} of an \ac{ide} refers to the
  \textit{domain} or field of knowledge under which the interface falls.
  That is, the field or fields for which the interface is primarily used,
  if any. Example values might include \texttt{General}, \texttt{Modeling},
  \texttt{Software}, \textit{etc}.

  \item[Accepted Values]

  \begin{AlignedDesc}
    \item[General] This \ac{ide} is general purpose. It can be applied to
    or used within any number of different fields to the same effect.
    \item[\textellipsis] Any custom field is allowed.
  \end{AlignedDesc}

  \item[Scoring] This variable is neutral. Domain is determined simply by
  the purpose of the \ac{ide} and has no meaningful impact on usability.

\end{AlignedDesc}
%}}}

%{{{ Subsubsection: Skill Level
\subsubsection{Skill Level}
\label{subsubsec:skill}

\begin{AlignedDesc}
  \item[Abbreviation] \texttt{Skill}

  \item[Variable Type] Nominal

  \item[Description] The \textbf{Skill Level} of an \ac{ide} describes what
  skill level within the given \textbf{Domain} is expected of its users.

  \item[Accepted Values]

  \begin{AlignedDesc}
    \item[Novice] The user is expected to have little to no experience in
    the domain. The \ac{ide} is designed to be user friendly and welcoming
    to new users with no former domain knowledge.
    \item[Intermediate] The user is expected to have a medium level of
    experience in the domain. The \ac{ide} may offer some advanced
    functionality, but still be welcoming to newer users.
    \item[Expert] The user is expected to have a high level of experience
    in the domain. The \ac{ide} offers highly sophisticated functionality
    that the user is expected to understand without further instruction.
    \item[General] The \ac{ide} offers a powerful set of advanced
    functionality while maintaining an overall level of accessibility and
    ease of use.
  \end{AlignedDesc}

\end{AlignedDesc}
%}}}

%}}}

% {{{ subsection: Chrome [incomplete]
\subsection{Chrome}
\label{subsec:chrome}

The \textbf{Chrome} of an \ac{ide} is the sum total of user interface
components external to the workspace. This includes every tool, menu,
button, or other user interface component not contained within the active
workspace area.

%{{{ Subsubsection: Popular Features
\subsubsection{Popular Features}
\label{subsubsec:features}

\begin{AlignedDesc}
  \item[Abbreviation] \texttt{Features}

  \item[Variable Type] Compound

  \item[Description] Many \acp{ide} provide support for a set of common
  operations.~\cite{murphy2006} defines a list of the top 10 most
  frequently used \ac{ide} features, which are reproduced here as boolean
  sub-variables. Each sub-variable indicates whether the \ac{ide} in
  question supports the use of that particular feature.

  \item[Components]

  \begin{AlignedDesc}
    \item[Delete] Delete a syntactic element in the workspace
    \item[Save] Save and export a model to a storage media
    \item[Paste] Duplicate a syntactic element in the workspace from the
    paste buffer
    \item[Content Assist] Provide suggestions or completion for elements
    \item[Copy] Place syntactic element from the workspace into paste
    buffer
    \item[Undo] Undo the user's most recent action
    \item[Cut] Place a syntactic element from the workspace into paste
    buffer, and remove from workspace
    \item[Refresh] Load contents of workspace and interface dashboard
    elements from storage media and update display if necessary
    \item[Show View] Open and display a new tool in the interface
  \end{AlignedDesc}

\end{AlignedDesc}
%}}}

%{{{ Subsubsection: Context Sensitive Tools
\subsubsection{Context Sensitive Tools}
\label{subsubsec:context}

\begin{AlignedDesc}
  \item[Abbreviation] \texttt{Context}

  \item[Variable Type] Boolean

  \item[Description] Any interface component which changes visibly or is
  generated anew depending on the context of selected elements within the
  workspace is a \textbf{Context Sensitive Tool}.

  \item[Accepted Values]

  \begin{AlignedDesc}
    \item[Yes] The \ac{ide} supports context sensitive tools.
    \item[No] The \ac{ide} does not support context sensitive tools.
  \end{AlignedDesc}

\end{AlignedDesc}
%}}}

%{{{ Subsubsection: Degree of Interface Visual Richness
\subsubsection{Degree of Interface Visual Richness}
\label{subsubsec:toolrichness}

\begin{AlignedDesc}
  \item[Abbreviation] \texttt{ToolRichness}

  \item[Variable Type] Compound

  \item[Description] The \textbf{Degree of Interface Visual Richness}
  describes the extent to which an interface utilizes visual variables to
  increase visual discriminability of available tools. The component
  variables measured are a composite of several authors' research,
  including elements from visual language design. Each component variable
  is a boolean measure determining whether or not the described visual
  variable is used in the interface to distinguish available tools.

  \item[Components]

  \begin{AlignedDesc}
    \item[Icons] Images contained in a border of a standard size and shape
    represent distinct actions or tools.~\cite{costagliola2002,moody2009}
    \item[Shape] Distinct shapes indicate different tools or classes of
    tool.~\cite{moody2009}
    \item[Size] Different tool sizes indicate distinct tools or classes
    of tool.~\cite{moody2009}
    \item[Color] Color is used to indicate distinct tools or classes of
    tool.~\cite{moody2009}
    \item[Text] Text (or typographic variation) is used to identify or
    distinguish tools or classes of tool.~\cite{moody2009}
    \item[Organizational Coherence] Components with related purpose are
    visually grouped in the interface.~\cite{constantine1996}
    \item[Texture] Shading or shadows are used to modify tools or to
    distinguish between distinct tools or classes of tool.~\cite{moody2009}
    \item[Brightness] The brightness of a color (\ie{} its perceived
    luminosity) is used to indicate a difference between tools or classes
    of tool.~\cite{moody2009}
  \end{AlignedDesc}

\end{AlignedDesc}
%}}}

%{{{ Subsubsection: Multiplicity of Perspectives
\subsubsection{Multiplicity of Perspectives}
\label{subsubsec:perspectives}

\begin{AlignedDesc}
  \item[Abbreviation] \texttt{Perspectives}

  \item[Variable Type] Ratio

  \item[Description] The number of available predefined interface
  perspectives available to the user. A perspective is defined as a visual
  configuration of the available tools in the interface chrome and elements
  in the workspace for the purposes of accomplishing a distinct task by
  means of a distinct process.

  \item[Range] $[1 , \infty)$

\end{AlignedDesc}
%}}}

%{{{ Subsubsection: Object Properties Window
\subsubsection{Object Properties Window}
\label{subsubsec:properties}

\begin{AlignedDesc}
  \item[Abbreviation] \texttt{Properties}

  \item[Variable Type] Nominal

  \item[Description] The \textbf{Object Properties Window} is any interface
  component that displays the properties of an element in the workspace.
  This component generally allows modification of element properties as
  well.

  \item[Accepted Values]

  \begin{AlignedDesc}
    \item[Omnipresent] A single property dialog or window is always
    present. The contents may update contextually.
    \item[Manual] The system requires user interaction to present the
    properties window.
    \item[None] No properties window is ever presented for workspace
    elements.
  \end{AlignedDesc}

\end{AlignedDesc}
%}}}

%{{{ Subsubsection: Searchable Toolspace
\subsubsection{Searchable Toolspace}
\label{subsubsec:searchable}

\begin{AlignedDesc}
  \item[Abbreviation] \texttt{Searchable}

  \item[Variable Type] Boolean

  \item[Description] The \textit{toolspace} is the total set of available
  tools, components, or actions that the \ac{ide} offers to the user. If an
  \ac{ide} employs a \textbf{Searchable Toolspace}, it allows the user to
  search through this toolspace by name or keywords.

  \item[Accepted Values]

  \begin{AlignedDesc}
    \item[Yes] The \ac{ide} supports a searchable toolspace.
    \item[No] The \ac{ide} does not support a searchable toolspace.
  \end{AlignedDesc}

\end{AlignedDesc}
%}}}

%{{{ Subsubsection: Toolbar Styles [incomplete]
\subsubsection{Toolbar Styles}
\label{subsubsec:toolstyle}

\begin{AlignedDesc}
  \item[Abbreviation] \texttt{ToolStyle}

  \item[Variable Type]

  \item[Description]

  \item[Accepted Values]

  \begin{AlignedDesc}
    \item[\textellipsis]
  \end{AlignedDesc}

\end{AlignedDesc}
%}}}

%{{{ Subsubsection: Visual Clutter
\subsubsection{Visual Clutter}
\label{subsubsec:clutter}

\begin{AlignedDesc}
  \item[Abbreviation] \texttt{Clutter}

  \item[Variable Type] Ordinal

  \item[Description] \textbf{Visual clutter} is defined as the number
  and organization of tools available on the screen versus the amount of
  workspace provided by the \ac{ide}. If the \ac{ide} offers no method for
  tool organization or if there is an immense amount of tools visible at
  once, then the \ac{ide} is likely visually cluttered.

  \item[Accepted Values]

  \begin{AlignedDesc}
    \item[Low] There is a minimal and cohesive amount of tools present on
      the screen.
    \item[Medium] There is a somewhat organized and manageable amount of
    tools available on the screen.
    \item[High] There is an unorganized or overwhelming amount of tools
    taking up space on the screen.
  \end{AlignedDesc}

\end{AlignedDesc}
%}}}

%}}}

% {{{ subsection: Human Interface
\subsection{Human Interface}
\label{subsec:humaninterface}

\textbf{Human Interface} components of an \ac{ide} include aspects of the
software interface that affect how the human user interacts with the \ac{ide},
either mechanically (\eg{} through physical devices and media) or
mentally (\eg{} the mental load required of the user to operate the
\ac{ide}).

%{{{ Subsubsection: Essential Efficiency
\subsubsection{Essential Efficiency}
\label{subsubsec:eefficiency}

\begin{AlignedDesc}
  \item[Abbreviation] \texttt{EEfficiency}

  \item[Variable Type] Continuous

  \item[Description] The \textbf{Essential Efficiency} of an \ac{ide}
  measures the level to which the system automates tasks for the user. It
  is calculated as the ratio of the number of steps in a concrete use case
  to the number of steps in the representative essential use case (see
  Appendix~\ref{app:euc}).~\cite{constantine1996}
%
  \begin{align*}
    1 - \frac{\text{Number of steps in concrete use case}}
             {\text{Number of steps in essential use case}}
  \end{align*}
%
  Unlike the metrics for automation put forth in~\cite{Wei1998}, this
  measure does not require user experience reports, and can be measured
  through simple use case analysis.

  \item[Range] $[0, 1]$

  \item[Critical Values]
  \begin{AlignedDesc}
    \item[$1$] There are no steps required to complete the concrete use
    case. Not feasible in practice.
    \item[$0$] Indicates a one-to-one relationship between the number of
    steps to complete a concrete use case and the number of steps in the
    essential use case.
  \end{AlignedDesc}

\end{AlignedDesc}
%}}}

%{{{ Subsubsection: Interface Efficiency
\subsubsection{Interface Efficiency}
\label{subsubsec:iefficiency}

\begin{AlignedDesc}
  \item[Abbreviation] \texttt{IEfficiency}

  \item[Variable Type] Continuous

  \item[Description] The \textbf{Interface Efficiency} of an \ac{ide} is a
  concept related to the productivity of an interface. It measures the
  number of physical actions (including keystrokes, mouse clicks, and fine
  mouse movements) required of the operator to complete a task compared
  against the number of abstract steps in the representative essential use
  case:
%
  \begin{align*}
    1 - \frac{\text{Number of physical actions to complete use case}}
             {\text{Number of steps in essential use case}}
  \end{align*}
%
  This metric is different from the \textit{Essential Efficiency} proposed
  in~\cite{constantine1996} because it studies physical user actions
  instead of concrete task steps. Also note that this variable can take
  negative values, indicating that the number of physical actions required
  to complete an essential use case exceeds the number of abstract steps in
  the use case.

  \item[Range] $(-\infty, 1)$

  \item[Critical Values]
  \begin{AlignedDesc}
    \item[$1$] There are no physical actions required. Complete automation,
    not feasible in practice.
    \item[$0$] Indicates a one-to-one relationship between the number of
    physical actions to complete a use case and the number of steps in the
    essential use case.
    \item[$<0$] Indicates a larger number of physical actions than steps in
    the essential use case. The lower the value, the less the degree of
    automation provided by the interface.
  \end{AlignedDesc}

\end{AlignedDesc}
%}}}

%{{{ Subsubsection: Keyboard Use
\subsubsection{Keyboard Use}
\label{subsubsec:keyboard}

\begin{AlignedDesc}
  \item[Abbreviation] \texttt{Keyboard}

  \item[Variable Type] Ordinal

  \item[Description] \textbf{Keyboard Use} refers to the extent to which an
  \ac{ide} utilizes the use of a keyboard. This can range from a complete
  absence of any keyboard actions to providing certain actions which only a
  keyboard can perform.

  \item[Accepted Values]

  \begin{AlignedDesc}
    \item[None] The \ac{ide} does not offer the use of a keyboard.
    \item[Simple] The keyboard is used only for typing annotations,
      properties, or comments.
    \item[Optional] The option to use a keyboard for some actions is
    available, but these actions can also be performed with a mouse.
    \item[Required] There are some actions which can be only be performed
    with a keyboard.
  \end{AlignedDesc}

\end{AlignedDesc}
%}}}

%{{{ Subsubsection: Mode of Element Creation
\subsubsection{Mode of Element Creation}
\label{subsubsec:mode}

\begin{AlignedDesc}
  \item[Abbreviation] \texttt{Mode}

  \item[Variable Type] Nominal

  \item[Description] The process by which the user creates elements in the
  given \ac{ide}. The \textit{Drag n Drop} process refers to a single
  mouse press event followed by a dragging motion of the mouse and
  completed when the mouse button is released. The \textit{Point n Click}
  process utilizes a single mouse click to indicate a selection and
  followed with subsequent mouse clicks elsewhere to define placement.

  \item[Accepted Values]

  \begin{AlignedDesc}
    \item[Drag n Drop (1:1)] A Drag n Drop method is used to create
    elements. At most one element can be created per action.
    \item[Drag n Drop (1:n)] A Drag n Drop method is used to create
    elements. Multiple elements can possibly be created per action.
    \item[Point n Click (1:1)] A Pint n Click method is used to create
    elements. At most one element can be created per action.
    \item[Point n Click (1:n)] A Point n Click method is used to create
    elements. Multiple elements can possibly be created per action.
  \end{AlignedDesc}

\end{AlignedDesc}
%}}}

%{{{ Subsubsection: Tertiary Interface Devices
\subsubsection{Tertiary Interface Devices}
\label{subsubsec:devices}

\begin{AlignedDesc}
  \item[Abbreviation] \texttt{Devices}

  \item[Variable Type] Nominal

  \item[Description] \textbf{Tertiary Interface Devices} refers to any
  third-party human interface devices which can be used to interact with
  the interface. This could include music alteration devices such as
  microphones and MIDI keyboards as well as the integration of mobile
  devices.

  \item[Accepted Values]

  \begin{AlignedDesc}
    \item[None] The \ac{ide} does not offer any functionality for
    third-party interface devices.
    \item[\textellipsis] Any custom field is allowed.
  \end{AlignedDesc}

\end{AlignedDesc}
%}}}

%}}}

% {{{ subsection: Integration
\subsection{Integration}
\label{subsec:integration}

\textbf{Integration} is the manner with which the \ac{ide} integrates with
the visual language it supports. This includes any visual representation of
language syntax or semantics, as well as any tools to assist the user with
understanding the supported language.

%{{{ Subsubsection: Allowed Relations Indicated
\subsubsection{Allowed Relations Indicated}
\label{subsubsec:relations}

\begin{AlignedDesc}
  \item[Abbreviation] \texttt{Relations}

  \item[Variable Type] Boolean

  \item[Description] This refers to an \ac{ide}'s ability to emphasize
  possible syntactically correct connection points. This is often
  demonstrated with either the highlighting of allowed relations or the
  dimming of impossible relations.

  \item[Accepted Values]

  \begin{AlignedDesc}
    \item[Yes] The \ac{ide} emphasizes the allowed relations.
    \item[No] The \ac{ide} does not emphasize the allowed relations.
  \end{AlignedDesc}

\end{AlignedDesc}
%}}}

%{{{ Subsubsection: Output Generation Style
\subsubsection{Output Generation Style}
\label{subsubsec:output}

\begin{AlignedDesc}
  \item[Abbreviation] \texttt{Output}

  \item[Variable Type] Nominal

  \item[Description] \textbf{Output Generation Style} describes the mode
  with which the \ac{ide} renders and displays output to the user. It is a
  dual axis variable, measuring whether output is \textit{direct} or
  \textit{indirect} as well as \textit{live} or caused by a
  \textit{trigger}. \textit{Direct/indirect} describes whether the user directly
  modifies output or acts via a layer of abstraction (\eg{} a
  model). \textit{Live/trigger} describes whether output is generated and
  displayed live or after some event triggered by the user.

  \item[Accepted Values]

  \begin{AlignedDesc}
    \item[Direct Live] Any modifications directly alter the output, which
    is dynamically updated.
    \item[Indirect Live] Any modifications indirectly alter the output
    (\eg{} via a model), which is dynamically updated.
    \item[Direct Trigger] Any modifications directly alter the output, but
    a final compilation or execution process is required to complete the
    output.
    \item[Indirect Trigger] Any modifications indirectly alter the output
    (\eg{} via a model), but a final compilation or execution process is
    required to complete the output.
  \end{AlignedDesc}

\end{AlignedDesc}
%}}}

%{{{ Subsubsection: Syntax Enforcement
\subsubsection{Syntax Enforcement}
\label{subsubsec:syntax}

\begin{AlignedDesc}
  \item[Abbreviation] \texttt{Syntax}

  \item[Variable Type] Nominal

  \item[Description] \textbf{Syntax Enforcement} measures the mode with
  which the \ac{ide} enforces language syntax requirements.

  \item[Accepted Values]

  \begin{AlignedDesc}
    \item[Explicit] The \ac{ide} displays a message when the user creates a
    visual syntax error.
    \item[Implicit] The \ac{ide} incorporates features that prevent the
    user from creating visual syntax errors.
    \item[None] The \ac{ide} does not enforce visual syntax, or it is only
    enforced when explicitly checked (\eg{} compiled, executed, \etc).
  \end{AlignedDesc}

\end{AlignedDesc}
%}}}

%}}}

% {{{ subsection: Language Syntax
\subsection{Language Syntax}
\label{subsec:languagesyntax}

The \textbf{Language Syntax} variables describe properties of the supported
visual language syntax. While not strictly components of the \ac{ide}, they
are intimately tied to the overall style of the \ac{ide} and thus
included in this document.

%{{{ Subsubsection: Complexity Management
\subsubsection{Complexity Management}
\label{subsubsec:complexity}

\begin{AlignedDesc}
  \item[Abbreviation] \texttt{Complexity}

  \item[Variable Type] Nominal

  \item[Description] Any characteristics or features of the visual language
  that serve to reduce the complexity of that language. Reducing complexity
  refers specifically to decreasing the level of ``diagrammatic
  complexity'' while maintaining information transfer to the
  user.~\cite{moody2009}

  \item[Accepted Values]

  \begin{AlignedDesc}
    \item[Modularization] Larger systems within the language are divided
    into smaller subtasks.~\cite{moody2009}
    \item[Hierarchy] Different systems within the language are represented
    at different levels of detail.~\cite{moody2009}
    \item[None] The language makes no effort to separate components.
  \end{AlignedDesc}

\end{AlignedDesc}
%}}}

%{{{ Subsubsection: Connection Style
\subsubsection{Connection Style}
\label{subsubsec:connection}

\begin{AlignedDesc}
  \item[Abbreviation] \texttt{Connection}

  \item[Variable Type] Nominal

  \item[Description] A language's \textbf{Connection Style} refers to the
  manner that element connections are displayed. This dual axis variable
  measures connections as \textit{overlapping/linked} as well as
  connection sources as \textit{point/region} based. The former axis
  describes the visual representation of connections, while the latter
  describes how links are connected. If the supported language does not
  register on these axes, it is not ``Connection-based''~\cite{costagliola2002}
  and must be \textit{geometric}.

  \item[Accepted Values]

  \begin{AlignedDesc}
    \item[Overlapping Points] The elements are connected by positioning
    specific points to overlap.~\cite{costagliola2002}
    \item[Overlapping Regions] The elements are connected by positioning
    specific regions to overlap.~\cite{costagliola2002}
    \item[Linked Points] The elements are connected by linking together
    distinct points on each element.~\cite{costagliola2002}
    \item[Linked Regions] The elements are connected by linking together
    distinct regions on each element.~\cite{costagliola2002}
    \item[Geometric] The elements are connected via geometric placement.~\cite{costagliola2002}
  \end{AlignedDesc}

\end{AlignedDesc}
%}}}

%{{{ Subsubsection: Degree of Language Visual Richness
\subsubsection{Degree of Language Visual Richness}
\label{subsubsec:languagerichness}

\begin{AlignedDesc}
  \item[Abbreviation] \texttt{LanguageRichness}

  \item[Variable Type] Compound

  \item[Description]

  \item[Accepted Values]

  \begin{AlignedDesc}
    \item[Icons] Images contained in the elements are used to represent
    different elements.~\cite{costagliola2002,moody2009}
    \item[Shape] Distinct shapes indicate different elements or classes of
    element.~\cite{moody2009}
    \item[Size] Different element sizes indicate distinct elements or
    classes of element.~\cite{moody2009}
    \item[Color] Color is used to indicate distinct elements or classes of
    element.~\cite{moody2009}
    \item[Text] Text (or typographic variation) is used to identify or
    distinguish elements or classes of element.~\cite{moody2009}
    \item[Texture] Shading or shadows are used to modify elements or to
    distinguish between distinct elements or classes of
    element.~\cite{moody2009}
    \item[Brightness] The brightness of a color (\ie{} its perceived
    luminosity) is used to indicate a difference between elements or
    classes of element.~\cite{moody2009}
    \item[Orientation] Rotation of an element indicates differences between
    distinct elements or classes of element.~\cite{moody2009}
    \item[Horizontal Position] The horizontal location of an element
    indicates differences between elements.~\cite{moody2009}
    \item[Vertical Position] The vertical location of an element
    indicates differences between elements.~\cite{moody2009}
  \end{AlignedDesc}

\end{AlignedDesc}
%}}}

%}}}

