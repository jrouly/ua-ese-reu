\section{Results}
\label{sec:results}


\paragraph{Alice3} Designed primarily as an educational programming
environment, Alice3 alienates skilled users by expecting a lower level of
skill in its domain. It makes up for this loss in accessibility, however,
with its wide, almost universal, support of common \ac{ide} features as
well as context-sensitive tooling. Alice3 provides a medium level of visual
richness in its interface chrome, but boasts one of the highest essential
efficiency values. This efficiency value is achieved through the ``MIT
Scratch'' style of visual syntax. However, despite greatly reducing
operator mental load, Alice3's clunky design only manages a near-neutral
interface efficiency. The introduction of optional keybindings is a
redeeming factor for Alice3 and, like most educational programming
\acp{ide}, Alice3 includes simple modular complexity management and a
visually appealing level of language visual richness.

\paragraph{AToMPM}

\paragraph{AudioMulch}

\paragraph{Blender} Designed for a skilled target audience of experts in
the domain, it is no surprise that Blender supports most common \ac{ide}
features or provides context-sensitive tooling. Its high level of chrome
visual richness and the large number of perspectives available relative to
the average found in this study also contribute to a high quality
interface. However, its featurefulness leads directly to the second highest
observed value for visual clutter. Blender possess no particular efficiency
techniques, remaining around a perfect one-to-one relationship with the
measured essential use cases. Its heavy use of the keyboard reduces
accessibility to a wider audience, although the target skill level is
already a limiting factor. Finally, Blender provides modularization
complexity management through saving and duplication tools,along with one
of the most visually rich languages observed.

\paragraph{Cameleon}

\paragraph{\acl{emf}}

\paragraph{\acl{grc}}

\paragraph{Grasshopper 3D }

\paragraph{Max}

\paragraph{MetaEdit+}

\paragraph{MIT AppInventor}

\paragraph{MST Workshop}

\paragraph{Piet Creator}

\paragraph{Scratch}

\paragraph{Simulink}

\paragraph{Stencyl}

\paragraph{Tersus}

\paragraph{TouchDevelop}

\paragraph{UMLet}

\paragraph{Violet}

\paragraph{VisSim}

\paragraph{Visual Paradigm}

\paragraph{Visual Use Case}

\paragraph{WebRatio}

\paragraph{YAWL}

