\section{Results}
\label{sec:results}


\paragraph{Alice3} Designed primarily as an educational programming
environment, Alice3 alienates skilled users by expecting a lower level of
skill in its domain. It makes up for this loss in accessibility, however,
with its wide, almost universal, support of common \ac{ide} features as
well as context-sensitive tooling. Alice3 provides a medium level of visual
richness in its interface chrome, but boasts one of the highest essential
efficiency values. This efficiency value is achieved through the ``MIT
Scratch'' style of visual syntax. However, despite greatly reducing
operator mental load, Alice3's clunky design only manages a near-neutral
interface efficiency. The introduction of optional keybindings is a
redeeming factor for Alice3 and, like most educational programming
\acp{ide}, Alice3 includes simple modular complexity management and a
visually appealing level of language visual richness.

\paragraph{AToMPM}

\paragraph{AudioMulch} Although AudioMulch offers a wide array of tools and an in-depth interface ideal for professionals in the music industry, the overall complexity of the design greatly reduces the accessability for anyone else. It supports a large amount of popular \ac{ide} features, however, and offers equally high essential and interface efficiancy ratings.  This is visually assisted by a relation-highlighting feature, which also provides AudioMulch with an implicit syntax enforcement, both of which greatly aid the user in model creation. Unfortunately, there is no effort made to manage the high amount of complexity within the \ac{ide} and virtually all of the canvas elements look exactly the same, ultimately awarding AudioMulch with a low language visual richness score.

\paragraph{Blender} Designed for a skilled target audience of experts in
the domain, it is no surprise that Blender supports most common \ac{ide}
features or provides context-sensitive tooling. Its high level of chrome
visual richness and the large number of perspectives available relative to
the average found in this study also contribute to a high quality
interface. However, its featurefulness leads directly to the second highest
observed value for visual clutter. Blender possess no particular efficiency
techniques, remaining around a perfect one-to-one relationship with the
measured essential use cases. Its heavy use of the keyboard reduces
accessibility to a wider audience, although the target skill level is
already a limiting factor. Finally, Blender provides modularization
complexity management through saving and duplication tools,along with one
of the most visually rich languages observed.

\paragraph{Cameleon}

\paragraph{\acl{emf}}

\paragraph{\acl{grc}}

\paragraph{Grasshopper 3D }

\paragraph{Max}

\paragraph{MetaEdit+}

\paragraph{MIT AppInventor} Much like Alice3 and the other educational
interfaces, MIT's AppInventor is designed for an unskilled, novice
audience, reducing the breadth of target audience. By only supporting three
of the top ten popular features, AppInventor additionally alienates its
audience through nonconformity with expected standards. While colorful,
AppInventor's interface only supports four visual variables to distinguish
elements in the chrome, relying heavily on icons and text. AppInventor
combines the ``MIT Scratch'' style of visual syntax along with a Drag n
Drop element creation workspace, resulting in high levels of effective and
interface efficiency. Another artifact of the ``MIT Scratch'' style of
visual syntax is its implicit syntax enforcement, assisting the user by
preventing illegal structures. Finally, the high level of language visual
richness and modular complexity management scheme result in an overall
visually pleasant experience.

\paragraph{MST Workshop}

\paragraph{Piet Creator}

\paragraph{Scratch} One of the earliest educational interfaces present in
this study, Scratch has influenced a great deal of later \acp{ide},
including Stencyl, AppInventor, and Alice3 to name a few. Scratch is
specifically targeted toward a novice audience with a low skill level, and
only supports three of the top ten popular features. The interface employs
an average level of visual richness variables, and does not even define an
object properties window. The sparse, cluttered interface redeems itself
through high values of essential and interface efficiency. Implicit syntax
enforcement provides a safe environment for learning users, and the highest
observed level of language visual richness provides an engaging, visually
rich display. The focus on efficiency and visual richness works well in an
\ac{ide} designed for education by accelerating reinforcement and engaging
student attention.

\paragraph{Simulink}

\paragraph{Stencyl}

\paragraph{Tersus}

\paragraph{TouchDevelop}

\paragraph{UMLet}

\paragraph{Violet}

\paragraph{VisSim}

\paragraph{Visual Paradigm}

\paragraph{Visual Use Case}

\paragraph{WebRatio}

\paragraph{YAWL}

