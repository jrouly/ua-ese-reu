\section{Results}
\label{sec:results}


%{{{ Feature Evaluations

\subsection{Feature Evaluations}
\label{sec:feature_evaluations}

\subsubsection{Audience}

\paragraph{Domain} This variable is neutral. Domain is determined simply by
the purpose of the \ac{ide} and has no meaningful impact on usability.

\paragraph{Skill} Because we are focusing on accessibility, aiming towards
a general skill level would be ideal. Novice users wouldn't be confused or
intimidated by the design, and expert users wouldn't feel limited or held
back. However, in terms of accessibility, Novice is better than
intermediate, which is better than expert.


\subsubsection{Chrome}

\paragraph{Features} This is relatively NEUTRAL, in that necessary features
are often determined by the visual language. In general when dealing with
visual languages though, Save, Delete, Copy, Cut, Paste, and Undo are
somewhat important.

\paragraph{Context} This feature is POSITIVE. They would be expected by
non-programmers and further focus the intuitiveness of the IDE.

\paragraph{ToolRichness} This feature is POSITIVE. The more methods the IDE
uses to differentiate the tools, the easier it is for the user to discern
the different tools.

\paragraph{Perspectives} This feature is POSITIVE. A larger amount of
Perspectives is better than a smaller amount. Different perspectives can
allow different visualizations of the same thing, well-adhering to various
working styles that a wide range of users would have. They also support
modularization, making the IDE more manageable.

\paragraph{Properties} This feature is POSITIVE. Whether it is omnipresent
or manual is situational, however. An omnipresent properties window takes
up screen real estate, but it is definitely more convenient. The amount of
the screen already consumed should be taken into account when deciding
between omnipresent and manual.

\paragraph{Searchable} This feature is POSITIVE. Including the ability to
search for tools that the user knows exists reduces time spent looking
through the toolbars for the specific tool.

\paragraph{ToolStyle} This feature is NEUTRAL. The usefulness of each
toolbar style is dependent on the function of the IDE and the purpose of
each individual toolbar.

\paragraph{Clutter} Visual clutter is a NEGATIVE feature of an interface.
Having too much clutter results in interfaces that are difficult to
navigate and place too much load on the user to simply keep straight in
their minds. IDEs should definitely be expected to employ means of reducing
clutter and maintaining a level of organization and neat tool placement.


\subsubsection{Human Interface}

\paragraph{EEfficiency} This feature has a simple relationship with the
goal: the more EEfficient an interface, the less mental work the user needs
to perform.

\paragraph{IEfficiency} This feature has a simple relationship with the
goal: the more IEfficient an interface, the less physical work the user
needs to perform.

\paragraph{Keyboard} This feature is POSITIVE. Any presence of keyboard use
(simple, optional, or required) gives the user more ability to interact
with the IDE, especially given that keyboard user and hotkeys are far more
productive in general than mouse use alone. The exception to this rule is
IDEs in a touch-only environment - this is as productive as can be
expected.

\paragraph{Mode} This feature can have a bidirectional effect on the goal.
The most common mode, "Drag n Drop (1:1)", is standard and A-OK. The mode
"Point n Click (1:1)" is slightly less common than this, but just as
productive overall. A slightly more productive mode, "Point n Click (1:n)",
is less common but probably better good given that it allows for faster
creation of elements.

\paragraph{Devices} This feature is NEUTRAL. While being able to interface
with multiple devices might be beneficial at times, depending on the domain
it might not even be feasible. Cannot be expected as a feature of an IDE.


\subsubsection{Integration}

\paragraph{Relations} This feature is POSITIVE. While not having this
feature is not necessarily a bad thing, having it aids the user by quickly
indicating what actions are allowed to them.

\paragraph{Output} This feature is neutral and can't really have an effect
toward the goal. Direct / Indirect isn't a good/bad mapping. Live/Trigger
can also both be good or bad.

\paragraph{Syntax} This feature is POSITIVE. Implicit syntax enforcement is
the best, followed by explicit, followed by no syntax enforcement at all.
Implicit syntax reduces the amount of load placed on the user by reducing
the number of available actions to a small set of legal actions. Explicit
enforcement requires the user to go back and make changes if he/she creates
an error. Having no syntax enforcement at all is simply dangerous and does
not help the user. It's neutral, though, because syntax enforcement can't
really be entirely expected.


\subsubsection{Language Syntax}

\paragraph{Complexity} This feature is POSITIVE. Complexity is bad,
managing it is good. Tools for complexity management help reduce user
mental load.

\paragraph{Connection} This feature is NEUTRAL and has no real effect
toward the goal. It simply provides a classification scheme for the
language types.

\paragraph{LanguageRichness} Like ToolRichness, having a higher amount of
variables acting toward LanguageRichness is a good thing. A higher amount
of visual discriminability results in an easier time for the user to

%}}}


%{{{ IDE Evaluations

\subsection{\acs{ide} Evaluations}
\label{subsec:ide_evaluations}

\paragraph{Alice3}
\paragraph{AToMPM}
\paragraph{AudioMulch}
\paragraph{Blender}
\paragraph{Cameleon}
\paragraph{\ac{emf}}
\paragraph{\ac{grc}}
\paragraph{Grasshopper 3D }
\paragraph{Max}
\paragraph{MetaEdit+}
\paragraph{MIT AppInventor}
\paragraph{MST Workshop}
\paragraph{Piet Creator}
\paragraph{Scratch}
\paragraph{Simulink}
\paragraph{Stencyl}
\paragraph{Tersus}
\paragraph{TouchDevelop}
\paragraph{UMLet}
\paragraph{Violet}
\paragraph{VisSim}
\paragraph{Visual Paradigm}
\paragraph{Visual Use Case}
\paragraph{WebRatio}
\paragraph{YAWL}

%}}}
