\section{Methods}
\label{sec:methods}

\subsection{\acl{mt}}
We took three unique screenshots of each \ac{ide} and darkened the
workspace area to remove attention from complex models rather than complex
\ac{ide} chrome. We then distributed these screenshots on Amazon.com
\ac{mt} \acp{hit}, requiring five unique workers for each
screenshot.

We ended up with 12 unique workers completing the tasks, identified
anonymously in Table~\ref{table:mturk_workers}.
%See
%\begin{verbatim}    cat Batch_1613411_batch_results.csv | sort | \
%        head -n 375 | awk -F "," '{print $1}' |  \
%        uniq -c\end{verbatim}
%for details.
On average, each worker spent about one minute studying an
image of the \ac{ide} and received \$0.02 for each task completed for an
effective hourly rate of \$1.00.

\noindent
\begin{table}[!htb]\centering
\ra{1.3}
\begin{tabular}{@{}cc@{}}\toprule
  \textbf{Worker Identifier} & \textbf{\ac{hit} Count} \\
  \midrule
  Worker A & 21 \\
  Worker B & 50 \\
  Worker C & 59 \\
  Worker D & 10 \\
  Worker E & 5 \\
  Worker F & 75 \\
  Worker G & 4 \\
  Worker H & 19 \\
  Worker I & 56 \\
  Worker J & 14 \\
  Worker K & 61 \\
  Worker L & 1 \\
  \bottomrule
\end{tabular}
\caption{Amazon \acs{mt} workers}
\label{table:mturk_workers}
\end{table}

We transform the resulting data into a set of five vectors, represented
similarly to the abbreviated Table~\ref{table:mturk_data} where numeric
values are ordinal data representing the variable \textbf{Visual Clutter}
on a scale from one to five, from low to high amounts of clutter.

\noindent
\begin{table}[!htb]\centering
\ra{1.3}
\begin{tabular}{@{}lccccc@{}}\toprule
  \textbf{\acs{ide}} & \textbf{raterA} & \textbf{raterB} &
  \textbf{raterC} & \textbf{raterD} & \textbf{raterE} \\
  \midrule
  alice1 & 3 & 2 & 5 & 4 & 4 \\
  alice2 & 4 & 2 & 4 & 4 & 5 \\
  alice3 & 2 & 5 & 4 & 1 & 2 \\
  appinventor1 & 2 & 5 & 4 & 5 & 4 \\
  appinventor2 & 1 & 3 & 3 & 4 & 3 \\
  appinventor3 & 4 & 3 & 3 & 3 & 2 \\
  \bottomrule
\end{tabular}
\caption{Amazon \acs{mt} data preview}
\label{table:mturk_data}
\end{table}

Because multiple reviewers were presented with each subject, we performed
an \ac{irr} measure to ensure agreement across reviewers. Using the R
statistical library \texttt{irr} we perform a two-way, agreement,
average-measure \ac{icc}. The result, \ac{icc} = 0.648, is within the
``good'' range of significance.~\cite{cicchetti1994,hallgren2012} This
\ac{icc} value indicates that the reviewers were, in general, in agreement
about their ranking of interface clutter. Note that there were more than
five participating reviewers total, despite five sets of reviews (\ie{} the
experimental design is not ``fully crossed''~\cite{hallgren2012}). We argue
that this is not significant, however, because the five sets of reviewers
are disjoint sets, acting as entirely independent actors.
